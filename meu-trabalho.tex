% ------------------------------------------------------------------------------
% Centro Federal de Educação Tecnológica de Minas Gerais - CEFET-MG
%
% Modelo de trabalho acadêmico em conformidade com as normas da ABNT
% (Tese de Doutorado, Dissertação de Mestrado ou Projeto de Qualificação)
%
% Projeto hospedado em: https://github.com/cfgnunes/latex-cefetmg
%
% Autores: Cristiano Nunes <cfgnunes@gmail.com>
%          Henrique Borges <henrique@cefetmg.br>
% ------------------------------------------------------------------------------

% Utiliza um modelo para impressão apenas no anverso
\documentclass[oneside]{cefetmg}

% Utiliza um modelo para impressão em frente (anverso) e verso
%\documentclass[twoside]{cefetmg}

% Utiliza um modelo que mostra a palavra "Capítulo" no início de cada capítulo
%\documentclass[oneside, nomecapitulo]{cefetmg}

% ------------------------------------------------------------------------------
% Pacotes utilizados
% ------------------------------------------------------------------------------
\usepackage[utf8]{inputenc}         % Codificação do documento
\usepackage[T1]{fontenc}            % Seleção de código de fonte
\usepackage{booktabs}               % Réguas horizontais em tabelas
\usepackage{color, colortbl}        % Controle de cores
\usepackage{graphicx}               % Inclusão de gráficos
\usepackage{indentfirst}            % Recua o primeiro parágrafo de cada seção
\usepackage{microtype}              % Melhora a justificação do documento
\usepackage{multirow, makecell}     % Tabelas com múltiplas linhas e colunas
\usepackage{verbatim}               % Exibe texto tal como escrito no documento
\usepackage{icomma}                 % Utiliza vírgulas em expressões matemáticas
\usepackage{subeqnarray}            % Subenumeração de equações
\usepackage{amsmath}                % Funções matemáticas
%\usepackage[charter]{mathdesign}   % Utiliza a fonte 'Charter BT' (excelente!)
\usepackage{newtxtext,newtxmath}    % Utiliza a fonte 'Times New Roman' (clone)
%\usepackage{palatino}              % Utiliza a fonte 'Palatino' (clone)
%\usepackage{amsfonts}              % Fontes e símbolos matemáticos
%\usepackage{latexsym}              % Mais símbolos matemáticos
%\usepackage{xfrac}                 % Escreve frações de forma compacta

% Utiliza a fonte 'Arial' (clone)
%\usepackage[scaled]{helvet}
%\renewcommand*\familydefault{\sfdefault}

% ------------------------------------------------------------------------------
% Configurações gerais
% ------------------------------------------------------------------------------

% Define o tamanho do recuo do parágrafo
\setlength{\parindent}{1.25cm}

% Define o espaçamento entre um parágrafo e outro
\setlength{\parskip}{0.0cm}

% Define a cor utilizada nos links do PDF
\definecolor{blue_link}{RGB}{0,80,128}

% Define as palavras-chave nos metadados do PDF
\hypersetup{pdfkeywords={%
    Palavra-chave 1, Palavra-chave 2, Palavra-chave 3, Palavra-chave 4.
}}

% Define a hifenização de palavras que não estão no dicionário
%\hyphenation{%
%    qua-dros-cha-ve
%    Kat-sa-gge-los
%}

% ------------------------------------------------------------------------------
% Inclui os arquivos do trabalho acadêmico
% ------------------------------------------------------------------------------

% Inclui o arquivo que contém os dados do trabalho acadêmico
% ------------------------------------------------------------------------------
% Dados do trabalho acadêmico
% ------------------------------------------------------------------------------

\titulo{Título do Trabalho}
%\title{Title in English}
\subtitulo{Subtítulo do trabalho}
\autor{Nome completo do autor}
\local{Alagoas}
\data{Outubro de 2021} % Normalmente se usa apenas mês e ano

% ------------------------------------------------------------------------------
% Natureza do trabalho acadêmico
% Use apenas uma das opções:
% - Tese (para Doutorado)
% - Dissertação (para Mestrado)
% - Projeto de Qualificação (para Mestrado ou Doutorado)
% - Trabalho de Conclusão de Curso (para Graduação)
% ------------------------------------------------------------------------------

\projeto{Projeto de Qualificação}

% ------------------------------------------------------------------------------
% Título acadêmico
% Use apenas uma das opções:
% - Doutor (para Doutorado)
% - Mestre (para Mestrado)
% - Bacharel (para Graduação)
% ------------------------------------------------------------------------------

\tituloAcademico{Doutor}

% ------------------------------------------------------------------------------
% Dados da instituição
% ------------------------------------------------------------------------------

\instituicao{INSTITUTO FEDERAL DE ALAGOAS - CAMPUS ARAPIRACA}
\programa{BACHARELADO EM SISTEMAS DE INFORMAÇÃO}

% ------------------------------------------------------------------------------
% Área de concentração e linha de pesquisa
% Observação: Indique o nome da área de concentração e da linha de pesquisa do
% programa de Pós-graduação nas quais este trabalho se insere. Se a natureza
% for Trabalho de Conclusão de Curso, deixe ambos os campos vazios.
% ------------------------------------------------------------------------------

%\areaconcentracao{Modelagem Matemática e Computacional.}
%\linhapesquisa{Métodos Matemáticos Aplicados.}

% ------------------------------------------------------------------------------
% Logomarca
% Observação: A logomarca da instituição deve ser colocada no mesmo diretório
% onde foi colocado o documento principal.
% O formato pode ser: pdf, eps, jpg ou png.
% ------------------------------------------------------------------------------

\logoinstituicao{3cm}{figuras/IF_vertical.png}

% ------------------------------------------------------------------------------
% Dados do(s) orientador(es)
% ------------------------------------------------------------------------------

\orientador{Prof. Dr. Nome do orientador}
%\orientador[Orientadora:]{Nome da orientadora}
\instOrientador{Instituição do orientador}

\coorientador{Prof. Dr. Nome do coorientador}
%\coorientador[Coorientadora:]{Nome da coorientadora}
\instCoorientador{Instituição do coorientador}

% ------------------------------------------------------------------------------
% Folha de Rosto
% ------------------------------------------------------------------------------

% Trabalho de Conclusão de Curso
%\preambulo{{\imprimirprojeto} apresentado ao Curso de Engenharia de Computação do Centro Federal de Educação Tecnológica de Minas Gerais, como requisito parcial para a obtenção do título de {\imprimirtituloAcademico} em Engenharia de Computação.}

% Projeto de qualificação de Mestrado ou Doutorado
\preambulo{{\imprimirprojeto} apresentado ao Programa de \mbox{Pós-graduação} em Modelagem Matemática e Computacional do Centro Federal de Educação Tecnológica de Minas Gerais, como requisito parcial para a obtenção do título de {\imprimirtituloAcademico} em Modelagem Matemática e Computacional.}

% Dissertação de Mestrado
%\preambulo{{\imprimirprojeto} apresentada ao Programa de \mbox{Pós-graduação} em Modelagem Matemática e Computacional do Centro Federal de Educação Tecnológica de Minas Gerais, como requisito parcial para a obtenção do título de {\imprimirtituloAcademico} em Modelagem Matemática e Computacional.}

% Tese de Doutorado
%\preambulo{{\imprimirprojeto} apresentada ao Programa de \mbox{Pós-graduação} em Modelagem Matemática e Computacional do Centro Federal de Educação Tecnológica de Minas Gerais, como requisito parcial para a obtenção do título de {\imprimirtituloAcademico} em Modelagem Matemática e Computacional.}


% Início do documento
\begin{document}

\pretextual % Define o estilo de página para os elementos pré-textuais

\imprimircapa                                           % Capa
\imprimirfolhaderosto*{}                                % Folha de rosto
\importarfichacatalografica{ficha-catalografica.pdf}    % Ficha catalográfica

% Use o comando abaixo para usar um modelo de folha de aprovação...
% ------------------------------------------------------------------------------
% Folha de aprovação
% ------------------------------------------------------------------------------

% Isto é um exemplo de Folha de aprovação, elemento obrigatório da NBR
% 14724/2011 (seção 4.2.1.3). Você pode utilizar este modelo até a aprovação
% do trabalho. Após isso, substitua todo o conteúdo deste arquivo por um
% documento digitalizado em PDF da página assinada pela banca.
%
% Para importar o PDF com a página assinada, basta usar o comando abaixo
% no documento principal deste trabalho (meu-trabalho.tex):
% \importarfolhadeaprovacao{folha_aprovacao.pdf}

\makeatletter
\begin{folhadeaprovacao}

    % Imprime o nome do autor
    \begin{center}
        \normalfont\large\scshape\textbf\imprimirautor
    \end{center}

    % Espaço vertical
    \vspace*{\stretch{2}}

    % Imprime o título do trabalho
    \begin{center}
        \ABNTEXchapterfont\LARGE\scshape\SingleSpacing{%
            \imprimirtitulo%
            \abntex@ifnotempty{\imprimirsubtitulo}{%
                {:\\}{\Large\imprimirsubtitulo}}%
        }
    \end{center}

    % Espaço vertical
    \vspace*{\stretch{0.5}}

    % Imprime o preâmbulo
    \abntex@ifnotempty{\imprimirpreambulo}{%
        \hspace{.3\textwidth}
        \hyphenpenalty=10000\hbadness=10000%
        \begin{minipage}{.6\textwidth}
            \imprimirpreambulo
        \end{minipage}
    }

    % Espaço vertical
    \vspace*{\stretch{0.5}}

    \begin{center}
        Trabalho aprovado. \imprimirlocal, 23 de outubro de 2021:
    \end{center}

    % Imprime os nomes dos membros
    \abntex@ifnotempty{\imprimirorientador}{\assinatura{\imprimirorientador \\ \imprimirinstOrientador}}
    \abntex@ifnotempty{\imprimircoorientador}{\assinatura{\imprimircoorientador \\ \imprimirinstCoorientador}}
    \assinatura{Prof. Dr. Nome do convidado 1 \\ Instituição do convidado 1}
    \assinatura{Prof. Dr. Nome do convidado 2 \\ Instituição do convidado 2}
    %\assinatura{Prof. Dr. Nome do convidado 3 \\ Instituição do convidado 3}
    %\assinatura{Prof. Dr. Nome do convidado 4 \\ Instituição do convidado 4}

    % Espaço vertical
    \vspace*{\stretch{0.5}}

    % Imprime local e data
    \begin{center}
        \normalfont\scshape{\imprimirlocal}\\
        \normalfont\scshape{\imprimirdata}
    \end{center}

\end{folhadeaprovacao}
\makeatother

% ...ou use o comando abaixo para importar uma folha digitalizada no formato PDF
%\importarfolhadeaprovacao{folha-aprovacao.pdf}

% ------------------------------------------------------------------------------
% Dedicatória
% ------------------------------------------------------------------------------

\begin{dedicatoria}

\end{dedicatoria}
          % Dedicatória

\begin{agradecimentos}

\end{agradecimentos}

       % Agradecimentos
% ------------------------------------------------------------------------------
% Epígrafe
% ------------------------------------------------------------------------------

\begin{epigrafe}

    \textit{``Por mim se vai à cidade das dores; por mim se vai à ininterrupta dor [...].
        Abandonai toda a esperança, ó vós que entrais!''}
    (Dante Alighieri, p. 17, inscrição à porta do Inferno)

\end{epigrafe}

% ------------------------------------------------------------------------------
% Edite o texto acima para inserir uma epígrafe de sua preferência
% ------------------------------------------------------------------------------
             % Epígrafe
% ------------------------------------------------------------------------------
% Resumo
% ------------------------------------------------------------------------------

\begin{resumo}

    Síntese do trabalho em texto cursivo contendo um único parágrafo.
    Para uma Tese de Doutorado o resumo deve conter, no máximo, 500 palavras.
    Para uma Dissertação de Mestrado o resumo deve conter, no máximo, 250 palavras.
    Para um Projeto de Qualificação o resumo deve conter, no máximo, 200 palavras.
    O resumo é a apresentação clara, concisa e seletiva do trabalho.
    No resumo deve-se incluir, preferencialmente, nesta ordem: breve introdução ao assunto do trabalho de pesquisa (incluindo motivação e justificativa para a realização deste trabalho), o que será feito no trabalho (objetivos), como ele será desenvolvido (metodologia), quais são os principais resultados obtidos ou esperados e a conclusão (compare os resultados com os da literatura e destaque as principais contribuições científicas do trabalho.

    \par\vspace{\baselineskip}

    \textbf{Palavras-chave}: Modelo Latex. Trabalho acadêmico monográfico. Normas ABNT.
\end{resumo}

% Para uma Tese de Doutorado o resumo deve conter, no máximo, 500 palavras.
% Para uma Dissertação de Mestrado o resumo deve conter, no máximo, 250 palavras.
% Para um Projeto de Qualificação o resumo deve conter, no máximo, 200 palavras.

% ------------------------------------------------------------------------------
% Escolha de 3 a 6 palavras ou termos que melhor representam seu trabalho.
% As palavras-chave são utilizadas para indexação. A letra inicial de cada
% palavra deve estar em maiúsculas. As palavras-chave são separadas por ponto.
% ------------------------------------------------------------------------------
            % Resumo
% ------------------------------------------------------------------------------
% Abstract
% ------------------------------------------------------------------------------

\begin{resumo}[Abstract]

    Translation of the abstract into english, possibly adapting or slightly changing the text in order to adjust it to the grammar of Standard English.
    Try to stay within the limit of: 500 word for a PhD Thesis;
    250 words for a Master Dissertation;
    200 words for a Qualifying Research Project.

    \par\vspace{\baselineskip}

    \textbf{Keywords}: Latex model. Academic work. ABNT standards. Another word.
\end{resumo}

% Para uma Tese de Doutorado o resumo deve conter, no máximo, 500 palavras.
% Para uma Dissertação de Mestrado o resumo deve conter, no máximo, 250 palavras.
% Para um Projeto de Qualificação o resumo deve conter, no máximo, 200 palavras.

% ------------------------------------------------------------------------------
% O restante da formatação deve manter-se igual ao do resumo em português,
% por exemplo, um único parágrafo.
% ------------------------------------------------------------------------------
            % Abstract
\imprimirlistafiguras                                   % Lista de figuras
\imprimirlistatabelas                                   % Lista de tabelas
\imprimirlistaquadros                                   % Lista de quadros
\imprimirlistaalgoritmos                                % Lista de algoritmos
% ------------------------------------------------------------------------------
% Lista de Siglas
% ------------------------------------------------------------------------------

\begin{siglas}
    \item[ABNT] Associação Brasileira de Normas Técnicas
    \item[DECOM] Departamento de Computação
\end{siglas}

% ------------------------------------------------------------------------------
% Edite a lista acima para definir as siglas utilizadas neste trabalho.
% ------------------------------------------------------------------------------
         % Lista de siglas
% ------------------------------------------------------------------------------
% Lista de Símbolos
% ------------------------------------------------------------------------------

\begin{simbolos}
    \item[$\alpha$]   Letra grega Alfa
    \item[$\beta$]    Letra grega Beta
    \item[$\gamma$]   Letra grega Gama
    \item[$\delta$]   Letra grega Delta
    \item[$\epsilon$] Letra grega Épsilon
    \item[$\zeta$]    Letra grega Zeta
    \item[$\eta$]     Letra grega Eta
    \item[$\theta$]   Letra grega Teta
    \item[$\iota$]    Letra grega Iota
    \item[$\kappa$]   Letra grega Kappa
    \item[$\lambda$]  Letra grega Lambda
    \item[$\mu$]      Letra grega Mi
    \item[$\nu$]      Letra grega Ni
    \item[$\xi$]      Letra grega Xi
    \item[$o$]        Letra grega Ômicron
    \item[$\pi$]      Letra grega Pi
    \item[$\rho$]     Letra grega Rô
    \item[$\sigma$]   Letra grega Sigma
    \item[$\tau$]     Letra grega Tau
    \item[$\upsilon$] Letra grega Upsilon
    \item[$\phi$]     Letra grega Fi
    \item[$\chi$]     Letra grega Chi
    \item[$\psi$]     Letra grega Psi
    \item[$\omega$]   Letra grega Ômega
\end{simbolos}

% ------------------------------------------------------------------------------
% Edite a lista acima para definir os símbolos utilizados neste trabalho.
% ------------------------------------------------------------------------------
       % Lista de símbolos
\imprimirsumario                                        % Sumário

\textual % Define o estilo de página para os elementos textuais

% ------------------------------------------------------------------------------
% Introdução
% ------------------------------------------------------------------------------

\chapter{Introdução}
\label{chap_introducao}

\begin{citacao}
Quantification is now more and more clearly seen not only as a neutral epistemic tool producing knowledge about the world, but also as a tool of power and government, a driving force in transforming the world \cite{didier2022}.
\end{citacao}
\begin{citacao}
Também em nossa época, como em todos os tempos, o fundamental no desenvolvimento do Direito não está no ato de legislar nem na jurisprudência ou na aplicação do Direito, mas na própria sociedade \cite{ehrlich1967fundamentos}. 
\end{citacao}
\begin{citacao}
Quando se faz jurimetria, estuda-se o Direito através das marcas que ele deixa na sociedade \cite{abj2022}.
\end{citacao}

Conforme \citeonline{didier2022}, supracitado, explana, a quantificação é uma ferramenta de poder e governo que produz conhecimentos sobre o mundo, bem como é uma força motriz que o transforma, afetando diversos aspectos da vida em sociedade como relacionamentos interpessoais, administração de organizações, avaliação de condutas, consolidação dos Estados modernos, evolução dos mercados financeiros, entre outros, de forma que tornou-se um expediente relevante para o desenvolvimento de outros campos do conhecimento que são umbilicalmente conectados a realidades sociais como a sociologia e o Direito \cite{camargo2021estudos}.

Como acima mencionado por \citeonline{ehrlich1967fundamentos}, o desenvolvimento do Direito baseia-se na sociedade, no conhecimento de suas realidades, conhecimento esse que pode ser obtido por meio da investigação dos modos de resolução de conflitos de suas associações sociais. Conflitos surgem do relacionamento mútuo de um conjunto de pessoas em uma certa ordem interna, definidora de direitos e deveres legais (regras jurídicas) entre elas, que, ao ser rompida, reafirma e/ou redefine essas regras através de meios institucionalizados, como o Estado \cite{ehrlich1967fundamentos}. Essa objetiva investigação para compreender o funcionamento de certa ordem jurídica é um movimento empírico que tem avançado rapidamente nas abordagens ao Direito visando a conhecer, por exemplo, os conflitos apresentados aos tribunais de justiça, a efetividade de uma lei nova, as relações de dominação, os contratos, os acordos, as associações, os testamentos, os casamentos, as decisões judiciais, as jurisprudências, entre outros \cite{ehrlich1967fundamentos} \cite{nunes2016jurimetria}.

Essas práticas concretas (contratos, sentenças, etc), que afetam associações sociais, demonstram um efetivo desenvolvimento do Direito ao concretizarem, por meio da solução de conflitos específicos, as normas jurídicas da sociedade. Esse processo de resolução de determinados problemas é influenciado não só pelas declarações de intenções de legisladores ou juristas por meio de, respectivamente, normas gerais ou individuais, mas também por valores e experiências pessoais, religião e tantos outros fatores que são objetos de estudo de um novo campo do conhecimento, a saber, a jurimetria \cite{nunes2016jurimetria}.

A jurimetria propõe-se a desenvolver o Direito estudando seu plano concreto e seus espaços institucionais de criação de normas visando a investigar, de forma indutiva, processos decisórios em que conflitos sociais são solucionados e, para tanto, utiliza a quantificação como ferramenta de abordagem a essa ciência a fim de reconhecer, caracterizar e organizar tendências e padrões jurisprudenciais, produzindo, assim, uma melhor compreensão do funcionamento, na prática, de determinada ordem jurídica \cite{nunes2016jurimetria}.

As marcas que o Direito deixa na sociedade referidas acima pela \citeonline{abj2022} não são exclusivamente as decisões judiciais e as legislações, mas são, também, outros fatos jurídicos como decisões administrativas, decretações de falências, formalização de contratos, recuperações judiciais de empresas, aumento do número de processos judiciais, quantidade de juízes por unidade federativa, etc. Fazer jurimetria, então, é aplicar métodos para estudar esses fatos jurídicos, porém não são quaisquer métodos; devem ser métodos quantitativos, a exemplo da estatística \cite{nunes2016jurimetria}.

Na obra brasileira seminal sobre essa nova área do conhecimento, \citeonline{nunes2016jurimetria} justifica apropriada a abordagem estatística a questões jurídicas explanando que o mundo social no qual o Direito fundamenta sua atuação e evolução não é determinista e, caso algum fenômeno de lá assim o fosse, o Direito só consegueria compreendê-lo por meio de observações empíricas e de inferências a partir destas. Dessarte, essa ferramenta quantitativa buscaria reconstituir um elemento de causalidade a fim de estudar (1) os diversos fatores - econômicos, valorativos, geográficos, etc - que condicionam a produção de normas, gerais ou individuais, e (2) as reações que essas normas provocam nas associações sociais \cite{nunes2016jurimetria}.

Nesse sentido, o objeto de estudo da jurimetria são as condutas, em função de normas jurídicas, de regulantes e regulados. Assim, tais normas não são propriamente o objeto de interesse, mas marcas que estabelecem momentos de decisões, registrando aspectos de seus sentidos e de suas motivações, que ajudam a entender o funcionamento de uma ordem jurídica por meio da apreensão estatística dos comportamentos coletivos em função dessas marcas \cite{nunes2016jurimetria}.

Mesmo que se consiga estudar as condutas individuais de cada sujeito de Direito integrado a certa sociedade, fazer inferências gerais a partir disso é temerário, pois os fenômenos regulares advindos dos comportamentos coletivos de agentes jurídicos são influenciados também pelo todo, pelo organismo social, que condiciona o comportamento de seus integrantes. Identificar essas influências, essas causas coletivas, pode auxiliar no entendimento da regularidade de certos padrões de conduta como fatos sociais e na previsão, por meio de modelos probabilísticos, das reações de um conjunto de pessoas em uma certa ordem jurídica frente a conflitos advindos de seu relacionamento mútuo. Devido a isso, a ferramenta quantitativa da estatística torna-se um método relevante à jurimetria para estudar, em função das normas jurídicas, condutas humanas não só individuais, mas também expressas em fatos sociais, que permeiam todo o Direito \cite{nunes2016jurimetria} \cite{ehrlich1967fundamentos}.

Verifica-se, nesse ponto, uma confluência entre as ciências sociais empíricas da sociologia e da jurimetria em que ambas utilizam-se da descrição de regularidades de comportamentos sociais para investigar as causas coletivas desses fenômenos padronizados \cite{van2006facts}. Uma das principais fontes que evidenciam comportamentos sociais para a jurimetria é o Judiciário, compreendido como um gerador de dados que apresentam o funcionamento completo da ordem jurídica, a exemplo do judiciário brasileiro, referido como \emph{pré-sal sociológico} por \citeonline{nunes2016jurimetria}, que pode ser compreendido como um quase-experimento devido a sua variabilidade e distribuição de materiais vastos para serem analisados através da sociologia e da estatística.

Outra aproximação dessas duas ciências sociais está na relevância que deram a práticas de quantificação. Enquanto a jurimetria as utilizou como ferramentas metodológicas de abordagem ao Direito, a sociologia as compreendeu como objeto de estudo devido a seus efeitos político-sociais que podem condicionar como indivíduos percebem a realidade social e se governam por essa percepção, a exemplo da confiança e da autoridade que indicadores, porcentagens e índices públicos, que são frequentemente ressignificados, costumam inspirar \cite{camargo2021estudos} \cite{daniel2013numeros}.

O entendimento da quantificação como ferramenta não só de medição mas também de coordenação inspirou diversas pesquisas acadêmicas, com destaque aos trabalhos do estatístico, sociólogo e historiador da ciência, Alain Desrosières. Ele explanou a quantificação como sendo a síntese das ações de convir e medir, retirando a falsa ideia de neutralidade, objetividade e tecnicidade dessa ferramenta \cite{desrosieres2013pour} \cite{camargo2021estudos}.

Ademais, os efeitos condicionantes do comportamento humano entre diferentes grupos sociais, advindos da quantificação como ferramenta de coordenação, ficam mais evidentes ao serem estudados pela sociologia, sendo revelados, assim, os modos de operação das relações de poder, a exemplo da objetivação, presente em qualquer forma de contagem, que aponta certos domínios não-comensuráveis que passam a ser convencionados pelos números, dotando, assim, de escala e dimensão aquilo que só podia ser referenciado qualitativamente, permitindo que influenciem na transformação da percepção coletiva, concorrendo para mudanças sociais \cite{camargo2021quantificaccao}.

Esses efeitos também ficam mais evidentes quando é o Estado que faz uso dessa ferramenta de coordenação. O governo de um Estado visa a solucionar problemas socialmente relevantes por meio da efetivação de políticas públicas. Mas a criação de uma política pública necessita objetivar tais problemas por meio de ferramentas quantitativas a fim de delimitar os domínios das ações governamentais e de dar legibilidade a esses problemas. Dessa forma, como a utilização da quantificação implica realizar definições operatórias convencionais para depois efetuar medições, então temos que essas investigações quantitativas oficiais irão refletir a agenda político-partidária do governo em questão e, portanto, os problemas socialmente relevantes serão objetivados segundo os interesses políticos dos governantes \cite{camargo2021estudos}.

Segundo \citeonline{nunes2016jurimetria}, discussões em torno de hipóteses de política pública devem se valer de instrumentos como a jurimetria para a realização de uma abordagem empírica, cautelosa e imparcial que levante os dados necessários à (1) objetivação de normas jurídicas que orientem a solução de certos conflitos de interesse caso sejam implementadas por meio de políticas públicas jurídicas e à (2) avaliação dos estímulos produzidos através de políticas públicas, se estes conseguiram condicionar comportamentos socialmente desejados. Esse autor também explana que essa tentativa de promover mudanças comportamentais nos sujeitos de Direito é um dos objetivos mediatos da ordem jurídica que, por meio de seus operadores que criam e aplicam normas, visa a propagar atitudes socialmente desejadas efetivando, assim, seu propósito como ferramenta de controle social \cite{nunes2016jurimetria}.

Observa-se, nessas propostas da jurimetria, uma convergência de seus conceitos com aqueles relacionados à quantificação sob uma ótica sociológica, a saber, de ferramenta de coordenação que tem ingerência sobre as vidas das pessoas a fim de condicionar determinados comportamentos. Esse relacionamento evidencia que eventuais pontes interdisciplinares entre os conceitos próprios desse novo campo do conhecimento e outras áreas do saber podem ser academicamente férteis.

Nesse sentido, a relevância da quantificação para a sociologia tornou-a um objeto de estudo a título próprio, a saber, a sociologia da quantificação, que é um campo do conhecimento que vem produzindo subsídios teóricos muito interessantes para evidenciar como fatores sociais, técnicos e políticos interagem na produção dos números que se dizem relevantes para a sociedade \cite{berman2018sociology}. Essa abordagem mostra que as estatísticas resultam de práticas sociais de categorização de dimensões da realidade e que esses números produzidos podem ser repolitizados de diversas maneiras, já que há escolhas e limitações implícitas em todos procedimentos quantitativos, de forma que a neutralidade e a imparcialidade são impossíveis e sempre há uma agenda normativa subjacente \cite{camargo2021quantificaccao}.

Assim, quando surgem novos campos do saber que se utilizam de expedientes quantitativos - pretensamente neutros, objetivos e imparciais - para abordar seus objetos de estudo, então capaz que seja razoável abordá-los também por meio da sociologia da quantificação a fim de melhor compreender essa arma poderosa, a quantificação, na produção dos conhecimentos e, consequentemente, das relações de poder nesses campos emergentes \cite{didier2013pour}. Todavia, são poucos os trabalhos acadêmicos que realizam essa abordagem sociológica.

Esta pesquisa, então, caminha para suprir essa lacuna acadêmica ao buscar compreender as principais propostas da jurimetria a partir das categorias e debates da sociologia da quantificação, visando a subsidiar melhores entendimentos sobre:
\begin{itemize}
	\item como práticas quantitativas estabelecem relações de coordenação e dominação ao produzirem conhecimentos, em que definem o que pode ser visto e o que deve permanecer oculto em certos momentos, implicando, assim, transformações da ordem político-social \cite{camargo2021estudos} \cite{mennicken2019s};
	\item como essas relações de poder influenciam na constituição das subjetividades ao implementarem novos modelos de racionalidade, como os baseados na eficiência e na utilidade, que transferem o risco para os indivíduos ao quantificarem seus comportamentos não-econômicos (decisões judiciais, criminalidade, vida familiar, etc) para analisá-los economicamente, de forma a remover dessa chave analítica princípios de proteção universal, de dignidade humana, de cooperação social, de empatia e de solidariedade \cite{camargo2021estudos} \cite{camargo2021quantificaccao};
	\item como os produtores de informações quantitativas por meio da jurimetria podem aumentar sua credibilidade técnico-científica e legitimidade político-social ao reconhecerem - e refletirem sobre - os efeitos e condicionantes advindos do uso dessa ferramenta nada imparcial \cite{camargo2021quantificaccao}.
\end{itemize}


\section{Justificativa}
\label{sec_motivacao}

A importância deste trabalho está em preencher uma lacuna na literatura acadêmica em relação a como resultados de discussões acerca da quantificação como objeto de estudo da sociologia podem ser utilizados para compreender propostas que quantificam fenômenos jurídicos, como a jurimetria.

\section{Definição do problema de pesquisa}
\label{sec_definicao_problema_pesquisa}

Como compreender as principais propostas da jurimetria a partir das categorias e debates da sociologia da quantificação?

\section{Objetivo geral}
Compreender a jurimetria a partir de suas relações com a sociologia da quantificação.

\section{Objetivos específicos}
\begin{itemize}
	\item Contextualizar como o ato de quantificar tornou-se um expediente relevante para a organização da sociedade e, portanto, para a sociologia como objeto a título próprio de estudo.
	\item Descrever contornos gerais de abordagens quantitativas do Direito e, mais especificamente, as principais propostas da jurimetria e seu impacto no Brasil e no mundo visando a analisá-la a partir de contribuições da sociologia da quantificação.
\end{itemize}

\section{Delimitação do tema}
\label{sec_contribuicoes}

A abordagem dessa pesquisa ao campo do conhecimento da quantificação, onde diferentes saberes se cruzam, gira em torno de um estudo das relações da sociologia da quantificação com a jurimetria visando a melhor compreender os impactos político-sociais que a quantificação aplicada ao Direito pode causar numa sociedade, chegando-se, assim, ao seguinte tema: ESTUDO DA JURIMETRIA NUMA ABORDAGEM SOCIOLÓGICA DA QUANTIFICAÇÃO.

\section{Organização do trabalho}
\label{sec_organizacao_trabalho}

Normalmente ao final da introdução é apresentada, em um ou dois parágrafos curtos, a organização do restante do trabalho acadêmico.
Deve-se dizer o quê será apresentado em cada um dos demais capítulos.

Segue um exemplo:

Este trabalho está organizado em capítulos, incluindo o presente.
No \autoref{chap_fundamentacao_teorica} são apresentados alguns dos principais conceitos necessários que fundamentam o desenvolvimento deste trabalho.
A \hyperref[chap_trabalhos_relacionados]{revisão bibliográfica} deste trabalho apresenta uma revisão dos principais estudos relacionados ao tema, descrevendo seus resultados e suas contribuições.
Por fim, no \autoref{chap_conclusao} são apresentadas as conclusões, bem como as perspectivas de trabalhos futuros.
               % Introdução
% ------------------------------------------------------------------------------
% Fundamentação Teórica
% ------------------------------------------------------------------------------

\chapter{Fundamentação Teórica}
\label{chapfundamentacaoteorica}

\section{Quantificação como objeto sociológico}

Conforme mencionado por Didier, a quantificação é uma ferramenta de poder e governança que produz conhecimento sobre o mundo, afetando diversos aspectos da vida social como as relações interpessoais, a gestão organizacional, a avaliação de condutas, a consolidação dos estados modernos e a evolução dos mercados financeiros. Tornou-se uma ferramenta relevante para o desenvolvimento de outras áreas do conhecimento intimamente ligadas às realidades sociais, como a sociologia e o direito.

\subsection{Quantificação segundo Alain Desrosières}

A quantificação, processo que envolve o ato de contar e medir, desempenha um papel significativo na sociedade. É uma ferramenta fundamental na aquisição e gestão do conhecimento, permitindo-nos dar sentido ao mundo que nos rodeia. Ao converter dados qualitativos em forma numérica, a quantificação nos permite categorizar, classificar e analisar vários fenômenos. Esse processo está profundamente enraizado em nossas estruturas sociais, influenciando nossa compreensão e interpretação da realidade \cite{boltanski2011critique}.

Um exemplo do uso da quantificação na classificação social pode ser encontrado no trabalho de Alain Desrosières, um renomado estatístico e sociólogo francês. No século XX, Desrosières empreendeu um estudo empírico e reflexivo das categorias socioprofissionais na França a partir de uma perspectiva estatística. Seu trabalho apresenta alguns dos papéis da quantificação na formação de estruturas e percepções sociais.

Desrosières descobriu que durante o processo de coleta de dados populacionais, os entrevistados relataram uma ampla gama de profissões e a tarefa do estatístico era atribuir a essas profissões um certo número de categorias que representassem grupos socialmente homogêneos \cite{desrosieres2016quantification}. Essa categorização partiu do pressuposto de certa afinidade entre pessoas de uma mesma categoria e a nomenclatura dessas categorias, segundo Desrosières, refletia o estado das lutas de classes e a configuração atual das relações de poder nas quais o estatístico estava imerso \cite{camargo2009sociologia}.

no
O trabalho de Desrosières desafia o entendimento tradicional das estatísticas como objetivas e imparciais ao destacar as dimensões sociais e políticas das práticas estatísticas, argumentando que esses processos estão profundamente enraizados em contextos sociais, históricos e epistemológicos. Para ele a compreensão das categorias na estatística reside na constante oscilação entre dois pólos opostos, mas complementares: o convencional e o real. A dimensão convencional refere-se aos aspectos socialmente construídos das categorias, enquanto a dimensão real diz respeito aos seus aspectos tangíveis, mensuráveis. Essa perspectiva ressalta a relação dialética entre a construção social das categorias e sua validação empírica, destacando a complexa interação entre quantificação, sociedade e poder. Esse trabalho de Desrosières apresenta um ponto de vista mais crítico por meio do qual é possível a compreensão do papel e do impacto da quantificação na sociedade. Além disso, essa obra ressalta a importância de reconhecermos as dimensões sociais e políticas das práticas estatísticas a fim de melhor compreendermos o mundo através de análises mais críticas aos números e categorias que nos são dados.

\subsection{Desenvolvimento da quantificação na sociedade}

As origens da quantificação remontam à Idade Média, quando o poder do príncipe baseava-se na propriedade de territórios e o domínio dessa ferramenta para administração dos recursos do príncipe era mister para a manutenção de seu poderio. Desde então a quantificação vem desenvolvendo-se na história da humanidade por meio da sua relação com a categorização e a classificação, pois quantificar aos poucos foi deixando de ser somente medir e expressar algo em números, a exemplo da sua aplicação na sociologia, permitindo mensurar, comparar e analisar vários aspectos da sociedade \cite{camargo2021social}. Ao contrário da aritmética, na quantificação, seu uso envolve não apenas números, mas também pessoas, respostas a questionários ou qualquer outra entidade no mundo. Isso destaca a contribuição das práticas quantitativas para a produção de conhecimento. Um exemplo do seu desenvolvimento histórico é o slogan do século XIX \emph{Um homem, um voto}. Esse slogan, que era um apelo à igualdade, excluía as mulheres pelas regras da convenção da época sobre quem poderia compor a categoria eleitora \cite{berman2018sociology}. Esse exemplo ilustra que a quantificação não é apenas um exercício mecânico ou matemático, mas uma complexa interação de convenções, classificações e medições. O processo de tradução de fenômenos sociais complexos em dados numéricos envolve os dois supracitados conceitos-chave: classificação e categorização. A categorização é a atribuição de uma entidade a uma categoria, enquanto a classificação envolve a atribuição de uma entidade a uma categoria e a avaliação da entidade relacionando-a a uma classe \cite{mennicken2019s}. Compreender esses significados e a diferenciação entre classificação e categorização é crucial para entender o conceito de quantificação no campo da sociologia. Ademais, a própria sociologia pode ser usada para analisar os processos centrais de quantificação e seus pré-requisitos socioepistemológicos, todavia o primeiro contato da quantificação com a sociologia pautou-se no desempenho de papéis significativos na compreensão dos fenômenos sociais, na avaliação de políticas e intervenções e na análise da organização social. Eventualmente o papel da quantificação na sociedade foi evoluindo bem como seu uso para servir como ferramenta de poder e governança por estar profundamente enraizada nos processos sociais e políticos e por moldar nossas percepções da realidade e influenciar a forma como percebemos e entendemos o mundo \cite{camargo2021social}.

Então, a quantificação na sociologia é um processo complexo que envolve a interação de convenções, classificações e medições e é uma ferramenta que tem sido usada historicamente para exercer poder e influenciar a organização da sociedade. Dessarte, seu estudo é essencial para a compreensão dos fenômenos sociais e para o desenvolvimento de políticas e intervenções efetivas na sociedade.

\subsection{O papel da quantificação nos assuntos de Estado}

O ato de quantificar tornou-se um expediente significativo para organizar a sociedade e administrar os Estados \cite{desrosieres1998politics}. A ascensão dos números em assuntos de Estado pode ser rastreada até o século 18, marcando uma mudança no modelo tradicional de gestão familiar. Este modelo foi substituído pela noção de população como um recurso fundamental do poder do Estado. Desde então, os números tornaram-se importantes mediadores das tecnologias contemporâneas de governo, apoiando intervenções que visam o corpo social. Essas intervenções se concentram em processos biológicos, como nascimentos, mortes, estado de saúde, expectativa de vida e longevidade. Essa racionalização do poder como prática de governo, conhecida como governamentalidade, envolve um conjunto complexo de instituições, procedimentos, análises, cálculos e táticas que visam a população como foco principal de conhecimento e controle \cite{diaz2016convention}. A quantificação, nesse contexto, serve de prova numérica para descrever a realidade, ganhando assim legitimidade científica e social para sua ação sobre essa realidade como ferramenta do governo. Essa dupla dimensão das estatísticas como instrumento de prova e ferramenta de governo ressalta sua importância nos assuntos de Estado \cite{bruno2014statactivism}.

Ademais, elas moldam as políticas públicas e refletem as expectativas sociais de vários grupos que aspiram ser legitimados como membros da sociedade local a fim de tornarem-se visíveis para tais políticas, como as de proteção social. A quantificação de vários aspectos da sociedade também permite o monitoramento e a avaliação das ações governamentais por meio de indicadores econômicos, indicadores sociais, dados demográficos etc. Esses números servem como importantes mediadores de tecnologias contemporâneas de governo, fornecendo uma base para a tomada de decisões e implementação de políticas. No entanto, o uso de números em assuntos de Estado tem seus desafios e limitações. A confiança na quantificação pode levar a uma simplificação excessiva de realidades sociais complexas, e a legitimidade dos números pode ser contestada diante de dados ou interpretações conflitantes \cite{camargo2021social}. Além disso, o uso de números pode ser manipulado para atender a interesses particulares, levantando questões sobre a transparência e a responsabilidade das ações governamentais. 

Esse aumento de números em assuntos de Estado e seu uso como uma ferramenta de governança têm profundas implicações para a sociedade. Embora a quantificação forneça um meio de organizar a sociedade e administrar os Estados, ela também levanta questões importantes sobre a legitimidade, transparência e responsabilidade das ações governamentais \cite{berman2018sociology}. Como tal, uma compreensão crítica do papel da quantificação nos assuntos do Estado é essencial para uma tomada de decisão informada e uma governação eficaz.

\subsection{Estatísticas, sua dupla dimensão e a realidade}

Uma das roupagens da quantificação é a estatística, campo que possui diversas dimensões, como a cognitiva. A dimensão cognitiva da estatística é um conceito complexo que se relaciona com o processo de quantificar, a saber, o estabelecimento de categorias e classificações e a aplicação do princípio da equivalência. Tal dimensão dos sistemas estatísticos não se limita ao domínio estatal ou das ciências, mas permeia todos os aspectos da vida social, influenciando nossa compreensão e interpretação do mundo que nos cerca.

O processo de quantificação envolve uma série de etapas, incluindo o estabelecimento de classificações, categorizações e de convenções de equivalência, que servem de base para a estrutura cognitiva dos sistemas estatísticos. O princípio da equivalência, conforme conceituado por Desrosières, serve como a lógica subjacente a esses processos. Essa conceituação do princípio da equivalência destaca a alhures dupla dimensão das categorias, que oscilam entre o convencional e o real, ressaltando a complexidade inerente e o dinamismo dos sistemas estatísticos.

Seguindo no processo de quantificação, as medições de entidades devem ser precedidas por convenções sobre essas mesmas entidades. Essas convenções, baseadas no princípio da equivalência, servem como a mencionada lógica implícita sobre a qual se baseiam as categorias e classificações, demonstrando como a quantificação é criada e, por sua vez, cria o mundo.

Dessarte a quantificação não apenas reflete a realidade; ela a constrói ativamente. Isso é feito por meio da criação de regras convencionais, que atribuem entidades, consideradas equivalentes entre si, a categorias de padronização. Essas regras não são simplesmente espelhos da realidade, mas são construções ativas que constituem e transformam a realidade.

No entanto, é importante notar que os princípios de classificação lógica só geram valor para um determinado grupo de categorias. Essa limitação ressalta a necessidade de uma compreensão diferenciada das dimensões cognitivas da estatística, a saber, entender como o processo de quantificação e a aplicação do princípio da equivalência são influenciados por uma variedade de fatores, incluindo contextos sociais, culturais e políticos.

\subsection{Espaços de equivalências cognitivas}

Outro conceito importante explanado por Desrosières é o relacionado à estruturação de espaços de equivalências cognitivas, onde determina-se o alcance político e geográfico de categorias e classificações por meio da influência de delimitações históricas e de lutas de classes. Esses espaços, utilizados como pontos de referência para debates e discussões, têm sido fundamentais para o uso - anteriormente mencionado - da quantificação pelo Estado. Tal uso como ferramenta de governo por instituições ocorre através da fixação de equivalências - por meio de consensos - nesses espaços, materializando, assim, categorias e classificações. Essas instituições, de natureza abstrata, criam tais consensos por meio do já mencionado processo de fixação de categorias que estruturam ordenadamente espaços de equivalências cognitivas. Assim, a utilização da quantificação pelo Estado como ferramenta de governo ocorre por meio do estabelecimento de equivalências, sob a lógica de princípios de equivalência, nesses espaços cognitivos. A ênfase no termo equivalência nesses conceitos se deve à especificidade da linguagem de quantificação, que permite comparações, transferências, agregações e manipulações padronizadas por meio de cálculos matemáticos. As equivalências cognitivas visam melhorar a governança por meio da produção de informações que auxiliam na descrição, gestão e transformação de grupos sociais. Alcança-se isso através da criação de instrumentos de ação pública, como indicadores que compilam dados populacionais contados e classificados, fornecendo informações cruciais para o exercício do poder.

A política, a administração e a vida pública estão cada vez mais interligadas com a produção e divulgação de dados estatísticos. Organizações internacionais, organizações não-governamentais e atores privados desempenham papéis significativos nesse processo, muitas vezes com implicações políticas significativas \cite{desrosieres2008argument}. As regras para a produção de indicadores, rankings e metas passaram a fazer parte da lógica política, servindo de base para debates e decisões políticas. Essas avaliações quantificadas do desempenho público são usadas para legitimar ou desafiar a autoridade política. A chamada cultura de resultados atesta o papel fundamental que os números desempenham não apenas na medição de realidades objetivas, mas na construção de identidades sociais e das próprias realidades \cite{berman2018sociology}.

\subsection{Engajamento crítico à quantificação}

Classificar e categorizar não são atos neutros, mas estão profundamente enraizados na política, evidenciando a dinâmica de poder inerente que muitas vezes é negligenciada no processo de quantificação. As escolhas feitas nesse processo não são arbitrárias, mas são influenciadas por uma série de fatores, incluindo normas sociais, interesses políticos e contextos históricos. Esses fatores, por sua vez, moldam a maneira como entendemos e interagimos com o mundo ao nosso redor. Números e dados estatísticos não são apenas ferramentas neutras para entender a realidade; são também instrumentos que sustentam argumentos políticos e legitimam certas ordens sociais \cite{chun2021logic}. A utilização de números e dados estatísticos na política é uma prova do seu poder de formação da opinião pública e da política. Eles são usados para justificar decisões, alocar recursos e estabelecer prioridades, refletindo e reforçando as estruturas de poder existentes. As classificações produzidas pelo processo de quantificação não são meramente descritivas; elas refletem as lutas de classes dentro da sociedade. Elas são o produto da negociação e contestação entre diferentes atores sociais, cada um com seus próprios interesses e perspectivas \cite{berman2018sociology}. Isso destaca a construção social das estatísticas, que é um processo que depende de eventos históricos e forças sociais. As estatísticas não refletem simplesmente a realidade; eles contribuem para criá-la. Este é um ponto crucial para compreender a dimensão política da quantificação. 

Os números que usamos para descrever o mundo não representam apenas uma realidade objetiva; eles também moldam nossa percepção dessa realidade. Essa dupla dimensão revela a complexa interação entre estatísticas e quantificações como construtos sociais e o mundo que ambas produzem e são produzidas por \cite{desrosieres1998politics}. As estatísticas permitem a geração de conhecimento. Elas fornecem uma estrutura para entender e interpretar o mundo e, assim, permitem o exercício do poder. O conhecimento gerado pela estatística não é neutro; é moldado por essas relações de poder que fundamentam o processo de quantificação.

Todavia, esse conhecimento, por sua vez, pode ser usado não somente para reforçar as estruturas de poder existentes, mas também para desafiá-las. As relações de poder engendradas pela quantificação não são apenas um subproduto do processo; eles são parte integrante dele, pois o ato de quantificar é um ato de poder que envolve definir o que é importante, o que é mensurável e o que vale a pena conhecer. Esse processo é dinâmico e pode ser manipulado - conforme o engajamento crítico de cada ator - pelas ordens sociais e estruturas de poder dentro das quais ocorre \cite{espeland2008sociology}.

\subsection{Categorias como construtos sociais}

O Estado, embora seja um ator significativo, não é o único envolvido na produção e utilização de estatísticas. A geração de dados está se tornando cada vez mais um assunto privatizado, e as convenções que sustentam esse processo geralmente são obscuras e opacas. Essa obscuridade das convenções não é mera coincidência, mas um reflexo das complexas dimensões cognitivas da quantificação \cite{desrosieres1988categories}. A dimensão cognitiva - já explanada - da quantificação diz respeito a como a quantificação molda nossa compreensão do mundo. E isso atinge significativamente as estruturas cognitivas dos atores sociais quando tais processos acabam sendo naturalizados, em que os produtos dos procedimentos de quantificação são aceitos como inevitáveis.

Devido a isso é essencial entender esses processos e suas implicações para apreciar com criticidade o papel da quantificação na produção de conhecimento e na formação de nosso mundo. Novamente, os processos de quantificação não oferecem apenas um reflexo do mundo; transformam-no e reconfiguram-no de outra forma \cite{camargo2021estudos}. Esta transformação não é um simples detalhe técnico mas tem implicações históricas, políticas e sociológicas. Evidencia-se isso nos fenômenos sociais que podem ser pensados, expressos, definidos e quantificados de múltiplas formas, cada uma com seu próprio conjunto de divergências e pontos de vista por serem processos sociais que envolvem negociação, contestação e luta de poder entre diversos atores. Essas características contestam a noção da estatística como inevitável ou natural e corroboram sua capacidade de alterar as relações de poder afetando como os recursos, status, conhecimento e oportunidades são distribuídos \cite{didier2021estatativismo}.

Desafiar essa naturalização das estatísticas se dá por meio da compreensão de que elas são construtos sociais, a saber, que não são dadas, mas são construídas socialmente para refletir os interesses e o poder daqueles que as constroem. A exemplo, a construção de categorias como raça, etnia e classe social, e a subsequente quantificação dessas categorias, podem moldar os entendimentos sociais e influir nas relações de poder existentes.

Ademais, importante compreender o poder das instituições na definição das coisas e, portanto, na construção da realidade, pois elas desenvolvem estatísticas públicas com a pretensão subjacente de consolidar sua própria existência. Visam fortalecer a dependência dos indivíduos em relação a elas, na ideia de que somente elas seriam capazes de sanar conflitos advindos de divergências entre pontos de vista de atores sociais. Essa dependência das instituições é uma prova de seu papel na definição da realidade, no estabelecimento da ordem e na naturalização dos produtos dos procedimentos de quantificação.

\subsection{Mundo, realidade e quantificação}

O mundo, como o entendemos, representa o fluxo de eventos, experiências e contingências da existência humana. É uma entidade complexa e em constante mudança que talvez nunca possamos compreender ou controlar totalmente devido à sua imprevisibilidade inerente e às limitações de nossa capacidade humana \cite{mottaresisting}. Este conceito de mundo encapsula a totalidade das experiências humanas, tanto tangíveis quanto intangíveis, que estão constantemente mudando. A realidade, por outro lado, só pode capturar uma fração da totalidade do mundo. Isso se deve a limitações intrínsecas, como a capacidade cognitiva humana bem como vieses e incompletude dos dados disponíveis. O processo de quantificação, embora útil para dar sentido a certos aspectos do mundo, muitas vezes é insuficiente para captar toda a complexidade e riqueza das experiências vividas \cite{hirata2021operaccoes}. Essa discrepância entre quantificação, realidade e mundo é uma área sensível, pois pode levar a distorções e deturpações do mundo como o conhecemos. Na obra de \citeonline{boltanski2011critique}, a realidade representa os arranjos, classificações e avaliações institucionalizadas que procuram dar sentido a este mundo. Esses arranjos estão intrinsecamente ligados às estruturas de poder social e político, que muitas vezes ditam as regras e normas que regem nossa compreensão da realidade. Porém, essas instituições que produzem classificações não são infalíveis \cite{camargo2022estado}. Elas estão sujeitas a dar ensejo a discrepâncias e preconceitos, tornando-as vulneráveis a críticas e questionamentos. Os choques e contradições potenciais dentro desse processo de construção da realidade oportunizam a crítica. É por meio dessa tensão que a crítica surge como uma ferramenta essencial para revelar e desafiar as estruturas de dominação \cite{boltanski2011critique}.

A evidenciação dessas discrepâncias permite o questionamento da autoridade dessas instituições e desafiar a reivindicação dessas regras. A exemplo, considere o caso das estatísticas públicas de distribuição de renda de uma instituição estadual \cite{susen2014spirit}. Essas estatísticas, embora retratem uma realidade de distribuição equitativa de riqueza, podem não estar alinhadas com as experiências vividas pelos cidadãos. A discrepância entre estas estatísticas públicas e as experiências vividas pelos cidadãos pode revelar desigualdades sociais gritantes, desafiando a autoridade da instituição e a sua representação da realidade. A quantificação muitas vezes desempenha um papel proeminente na produção da autoridade dos fatos. No entanto, é fundamental enfrentar essa autoridade de forma crítica e reflexiva \cite{boltanski2011critique}. Devemos questionar como as estatísticas e os dados são manipulados a fim de verificar eventuais deturpações e usos para enganar e iludir. Essa reflexão crítica nos permite questionar a seletividade desses consensos institucionais, que muitas vezes resultam da influência de determinações históricas pautadas pelas lutas de classes.

\subsection{Quantificação e dominação}

A reação das instituições às críticas sufocando-as ou resistindo a elas é um fenômeno característico das tentativas da classe dominante em manter o controle. Essa resistência à crítica tem o potencial de levar a um tratamento injusto nos processos de quantificação, o que pode minar ainda mais a legitimidade dessas instituições \cite{keslassy2014isabelle}. Essas instituições muitas vezes afirmam, como já mencionado, ser o único e exclusivo padrão da verdade, condição que simboliza certa violência aos indivíduos. Esse posicionamento induzido os obriga a se alinhar no espaço social de acordo com essas regras, muitas vezes distorcidas em favor da classe dominante. Esta, em virtude de seu controle sobre essas instituições, é capaz de limitar as faculdades reflexivas e críticas do restante da população tornando mais fácil para a classe dominante a exploração dos oprimidos para obtenção de lucro \cite{starr1992social}. Essa exploração é muitas vezes mascarada pelo uso da quantificação, que pode sustentar distinções sociais opressivas e exacerbar as desigualdades sociais.

O uso de estatísticas na formulação de políticas pode ter impactos significativos sobre as desigualdades sociais, especialmente em regiões como a América Latina, onde as disparidades sociais são gritantes. Um exemplo histórico disso é a quantificação dos negros como 3/5 de uma pessoa nos censos americanos do século XIX, resultado de um consenso alcançado durante a Convenção Constitucional dos Estados Unidos de 1787.

Reforçando o que já foi explanado alhures, as estatísticas informam nosso pensamento e tomada de decisões cotidianas, lapidando nossas interações e relacionamentos sociais. Elas influenciam nossos valores e normas culturais, moldam nossas percepções e expectativas, nossas identidades e subjetividades e nosso senso do que é possível e desejável \cite{hacking1990taming}. Na sociedade capitalista contemporânea, as estatísticas desempenham um papel mister como instrumento governamental para reger a população ao ajustar a distribuição espacial dos indivíduos à acumulação de capital, ao crescimento de grupos e à distribuição diferencial de lucros. Porém, como também já mencionado, os efeitos da quantificação não se restringem a determinados grupos sociais, mas se espalham por toda a sociedade, influenciando as formas como percebemos e interagimos com o mundo e possibilitando a apropriação da quantificação como ferramenta para os movimentos sociais, permitindo-lhes contestar o domínio de instituições poderosas e questionar as realidades que essas instituições promovem \cite{camargo2021estudos}.

\subsection{Estatativismo}

Estatativismo, um termo cunhado por \citeonline{didier2021estatativismo}, encapsula o uso estratégico da quantificação pelos movimentos sociais para desafiar o domínio dessas instituições poderosas e questionar as realidades que promovem. Este conceito baseia-se na crença de que números e estatísticas não são neutros, mas sim ferramentas que podem ser usadas para defender ou desafiar as estruturas de poder existentes. 

O estatativismo é visto como uma forma de resistência contra a opressão e a injustiça social, tendo como objetivo último a emancipação política. É uma crítica ao status quo, um apelo à desconstrução de narrativas que muitas vezes são tidas como certas, e um apelo à criação de novas equivalências que melhor reflitam as complexidades do nosso mundo. Os estativistas aproveitam o poder da quantificação para criticar e desconstruir a realidade criada por meio dos números, permitindo assim o estabelecimento de novas equivalências. Eles usam a estatística não apenas como uma ferramenta de descrição, mas como uma arma para denunciar realidades e formalizar críticas para influenciar e reformar governos. Eles são participantes ativos que procuram intervir na realidade de forma a promover a justiça e a igualdade. Em sua essência, o estativismo é uma resposta à lacuna inevitável entre o mundo como ele é e a realidade institucionalizada que tentamos impor a ele. Enquanto o ato de quantificar permanece fundamental na organização das sociedades e na administração dos Estados, o estativismo propõe aproveitar esse poder em toda a sociedade, especialmente entre os oprimidos, e não deixá-lo exclusivamente nas mãos de instituições poderosas. É um chamado para democratizar o poder de quantificação, para torná-lo uma ferramenta que pode ser usada por todos para desafiar e remodelar o mundo.

O estatativismo prospera nas discrepâncias que existem nas instituições que produzem classificações. É nessas discrepâncias que os estativistas encontram espaço para questionar, desafiar e, finalmente, mudar as narrativas dominantes. Eles usam a quantificação como uma ferramenta de luta e resistência contra a opressão e a injustiça social, estabelecendo coletivos plurais que buscam contestar a realidade criada através dos números.

Reforçando de outro forma tal conceito que é complexo, o estatativismo se beneficia quando as temporalidades das classificações no espaço público não correspondem mais às temporalidades sociais. Esse desalinhamento cria oportunidades para os estativistas desafiarem as narrativas dominantes e proporem alternativas que reflitam melhor as realidades do mundo \cite{lara2019acesso}. Eles fazem uso de estatísticas como uma forma de ativismo político, por exemplo, no caso da comunidade LGBTQ em que a publicação do relatório Kinsey em 1948 revelou que a proporção de homens que tiveram relações exclusivamente homossexuais ao longo da vida era muito mais alta do que se pensava anteriormente. Essa revelação desafiou a narrativa dominante sobre a homossexualidade e abriu caminho para uma maior aceitação e direitos para a comunidade LGBTQ.

Então, o estatativismo revela a falibilidade, o domínio e a opressão inerentes às realidades institucionalizadas. O poder da quantificação, revelado por meio do estativismo, oferece uma ótica crítica para a compreensão dos assuntos do Estado e das práticas institucionais. Dessarte, o Estatativismo não é apenas sobre números; trata-se de usar esses números para criar um mundo mais justo e igualitário.

\subsection{Quantificação como objeto sociológico}

A discussão acerca do Estatativismo enseja uma abordagem crítica abrangente das dimensões políticas e cognitivas das estatísticas e quantificações. Essa crítica desafia a neutralidade e a objetividade frequentemente assumidas das estatísticas e quantificações e expõe seus preconceitos sociais, políticos e culturais inerentes. É essencial salientar que esses vieses não são meramente acidentais, mas muitas vezes estão embutidos nos próprios métodos e suposições que sustentam a coleta, análise e interpretação dos dados. A validade e a confiabilidade das estatísticas e dos dados também devem ser questionadas nessa crítica \cite{lowrey2019data}. Os métodos usados para coletar dados, as suposições feitas no processo e as interpretações extraídas dos dados devem ser todas examinadas. Isso não é para minar o valor das estatísticas e quantificações, mas sim para garantir seu bom uso e evitar abusos por meio de manipulações ou deturpações para servir a interesses ou agendas particulares \cite{espeland2008sociology}. Compreender essa abordagem crítica é fundamental para uma utilização responsável e ética dessa ferramenta tão poderosa.

A estatística, quando usada com responsabilidade, pode ajudar a desmistificar falsas impressões sobre a realidade. No entanto, elas também podem servir para reforçar equívocos e preconceitos existentes, especialmente quando são usadas sem uma compreensão crítica de suas limitações e potenciais armadilhas. O uso de números na esfera estatal, por exemplo, tem seu próprio conjunto de desafios e limitações \cite{sareen2020ethics}. Os críticos argumentam que uma confiança excessiva na quantificação pode levar a uma visão estreita e reducionista da realidade, resultando em uma simplificação excessiva de questões sociais complexas e na negligência de fatores importantes devido à confiança em números, a exemplo do uso de testes padronizados na educação, já que o complexo processo de aprendizagem é muitas vezes reduzido a uma única pontuação numérica, o que pode levar a uma simplificação excessiva do processo de aprendizagem e à exclusão de aspectos qualitativos importantes \cite{laevers1994innovative}.

Da mesma forma, a classificação e a aferição de crimes refletem certas interpretações da lei e da ordem e moldam nossa compreensão do crime e da criminalidade de maneiras que podem não captar totalmente a complexidade desses fenômenos. Há uma crítica crescente contra a autoridade dos fatos, particularmente em relação a como estatísticas e dados são usados para justificar e legitimar certas políticas e práticas, e para marginalizar e excluir outras. Essa crítica visa capacitar grupos sociais para compreender as estatísticas públicas de forma crítica e reflexiva como construções sociais ligadas ao poder, rompendo com os pressupostos do senso comum de que tais números são fatos objetivos \cite{sellar2018feel}.

Para tanto, a sociologia da quantificação surge como espaço de crítica e resistência ao envolver a investigação da manipulação e deturpação de estatísticas e dados, e a crescente privatização da coleta e análise de dados \cite{gillborn2018quantcrit}. Esse campo de estudo também destaca as limitações da quantificação, particularmente em relação ao potencial de uso indevido e abuso de dados.

\subsection{A sociologia da quantificação}

A sociologia da quantificação, um campo de estudo em rápida expansão, preocupa-se principalmente com as implicações sociais da quantificação, do processo de transformação de fenômenos sociais complexos em dados numéricos \cite{berman2018sociology}. Esse campo examina criticamente a produção e comunicação de números, incluindo gráficos como representações visuais de dados numéricos, em relação ao poder político, sociedade e questões clássicas de pesquisa sociológica, como desigualdade social, pluralidades de avaliação e coordenação, conflito e crítica, racionalização, divisão do trabalho e sua organização, e cognição social.

O papel da sociologia na compreensão da influência da quantificação é fundamental ao examinar criticamente os pontos fortes e fracos da quantificação e suas implicações para a compreensão dos fenômenos sociais \cite{camargo2021quantificaccao}. A sociologia da quantificação fornece um ponto de vista crítico através do qual podemos entender o papel da quantificação na formação de nossa compreensão da realidade social. Essa perspectiva sobre a quantificação ressalta suas dimensões construídas e suas dimensões reais, demonstrando como as estatísticas produzem e são produzidas.

Portanto, o exame dos processos sociais e das relações de poder que sustentam a produção e o uso de números pode esclarecer as diversas maneiras pelas quais a quantificação pode iluminar ou obscurecer as realidades sociais, revelando quanto distorcendo tais fatos.

\subsection{A relação entre sociologia da quantificação e justiça}

Reiterando, a sociologia da quantificação postula que a quantificação está longe de ser neutra ou objetiva, desafiando a visão convencional dos números como imparciais e neutros. Em vez disso, destaca as relações de poder ocultas subjacentes à produção de estatísticas e o papel da sociologia em descobrir essas relações. Essa perspectiva transformou a percepção da relação entre estatística e política, revelando que essas ferramentas não são apenas instrumentos neutros de aferição. Em vez disso, elas estão profundamente arraigadas no tecido social e político da sociedade servindo como instrumentos de poder e governança \cite{turnbull2022slow}.

A sociologia da quantificação também examina criticamente como os números e dados estatísticos direcionam as políticas públicas, a pesquisa acadêmica e nossas percepções da realidade. Ela desafia a objetividade percebida de números e estatísticas, argumentando que são construções sociais profundamente enraizadas em estruturas sociais e percepções humanas. Um debate importante nesse campo gira em torno do conceito de objetividade \cite{salais2012quantificação}. 

Embora a quantificação ofereça o potencial de objetividade e precisão, ela também pode ser influenciada por preconceitos, suposições e limitações. Como já dito, estatísticas e quantificações são frequentemente usadas por atores políticos para legitimar determinadas políticas ou posições, corroborando a dinâmica de poder inerente ao processo de quantificação. Dessarte a sociologia da quantificação nos desafia a questionar as suposições que sustentam a produção e uso de números \cite{turnbull2022slow}. Essa perspectiva crítica é importante para promover uma compreensão mais matizada do papel da quantificação na sociedade, todavia, apesar dos avanços significativos feitos neste campo, ainda existem lacunas significativas na literatura, principalmente no que tange suas implicações para a justiça social numa perspectiva de informar abordagens de quantificação mais equitativas e democráticas, contribuindo para uma sociedade mais justa e inclusiva.

Outro aspecto da relação entre quantificação e justiça resvala no conceito de objetividade. Enquanto alguns argumentam que a quantificação pode fornecer conhecimento objetivo sobre o mundo, outros afirmam que todo conhecimento é socialmente construído e, portanto, inerentemente subjetivo. Essa dicotomia entre objetividade e subjetividade é um tema central no discurso sobre a quantificação \cite{desrosieres2009real}. A quantificação, em sua busca pela objetividade, busca a igualdade de tratamento e a imparcialidade nos processos dos indivíduos. Isso se baseia na premissa de que os números, sendo desprovidos de viés pessoal, podem fornecer uma representação neutra e justa da realidade. No entanto, essa perspectiva é muitas vezes contestada pelo argumento de que a quantificação, em sua essência, é uma construção social moldada pelos contextos sociais, políticos e culturais em que foi produzida e utilizada, logo, seria inerentemente subjetiva podendo, assim, potencialmente perpetuar os preconceitos daqueles que a criaram \cite{desrosieres2009real}.

Além disso, quando utilizada na esfera pública, a governança pela quantificação vê a diversidade das práticas sociais a partir de uma pretensa objetividade visando criptografar o mundo em representações padronizadas desconectadas das experiências individuais. Essa abordagem, embora aparentemente imparcial, muitas vezes desconsidera possíveis iniqüidades de tratamento da pessoa humana em processos de quantificação muitas vezes opacos em relação às suas regras de elaboração, observação e interpretação dos números. Essa falta de transparência pode prejudicar a legitimidade das instituições governamentais, especialmente quando indivíduos que acreditam ter sofrido desrespeito à sua dignidade encontram dificuldades para contestar esses instrumentos.

A crítica da quantificação vai além de sua falta de transparência. Envolve reconhecer as limitações dos fatos objetivos nos processos de quantificação. Essa crítica não visa rejeitar a quantificação, mas defender uma abordagem abrangente que incorpore outras metodologias de pesquisa e formas de quantificação mais justas para aumentar sua legitimidade na sociedade \cite{salais2016quantificação}. Objetiva também entender como a quantificação pode contribuir ou prejudicar a busca da justiça social \cite{alonso1987politics}.

\subsection{Direito como objeto quantitativo}

A quantificação, o processo de medir e expressar algo em números, tem sido cada vez mais utilizada no sistema de justiça, particularmente no âmbito das diretrizes de condenação \cite{salais2016quantification}. No entanto, essa prática pode prejudicar a busca pela justiça social, pois muitas vezes falha em dar conta das nuances e complexidades de casos individuais. No final do século 20, uma legislação foi aprovada nos Estados Unidos para padronizar as sentenças criminais. Essa foi uma tentativa de aumentar a objetividade e neutralidade no sistema de justiça americano. No entanto, a implementação desse sistema foi repleta de desafios. Após vinte anos de experiência, os juristas americanos chegaram ao consenso de que a aplicação obrigatória de um sistema de diretrizes de condenação foi um fracasso \cite{espeland2008sociology}. A condenação mecânica provou ser muito trabalhosa na prática, e a ignorância das idiossincrasias de casos concretos frequentemente produzia sentenças irracionais e injustas.

A ideologia neoliberal dominante, então, submeteu perigosamente a justiça e a sociedade estadunidense à ordem do cálculo, permitindo a reificação das pessoas por meio de sua história criminal para admitir a quantificação como norma de seu julgamento e como instrumento de seu controle. O impacto dessas ideologias dominantes na percepção da justiça e do Direito não pode ser subestimado. É crucial que cientistas e profissionais do direito questionem e critiquem o uso da quantificação na busca da justiça. Esse exame crítico da quantificação pode fornecer informações valiosas sobre como as ideologias dominantes moldam a percepção da justiça e do Direito. Entretanto, não basta apenas criticar o atual sistema mas também é necessário defender formas mais justas de quantificação para aumentar sua legitimidade na sociedade \cite{camargo2022estado}. O exemplo supracitado de padronização de sentenças criminais evidenciou como pode-se exacerbar as desigualdades sociais existentes ao afetar desproporcionalmente as comunidades marginalizadas \cite{lynch2019narrative}. Esse caso claro de como a quantificação pode minar a justiça, em vez de promovê-la, enseja o debate de diversos aspectos que serão trabalhados no próximo capítulo.

The increasing use of quantitative methods, such as statistical analyses and predictive algorithms, in legal contexts is shaping legal outcomes by influencing how cases are argued, decided, and justified \cite{101017s0003975609000150,ribeiro2021quantification,silva2023role}. Legal technologies that analyze court records and predict lawsuit outcomes rely on the availability of reliable data about past court behavior, which can enhance transparency and efficiency in the legal system but also raise concerns about potential biases and ethical implications of relying on algorithms for legal decision-making \cite{silva2023role}. Quantification in the legal context is evident in the use of quantitative guidelines for sentencing, which aim to standardize and objectify the sentencing process but are reconstituted through narrative forms by legal actors to fit their visions of justice, highlighting the limitations of quantification in capturing the full complexity of legal decision-making \cite{101111lsi12334}.

The use of quantitative methods in law can impact social realities by legitimizing legal decisions and shaping public perceptions of justice and fairness, but the reliance on quantitative data can also mask subjective judgments and biases inherent in legal decision-making, making it harder to challenge unjust outcomes and potentially undermining public trust in the legal system \cite{10.5040/9781350220645,10.1080/07329113.2015.1046739}. Quantification provides stakeholders in the adjudication process with rhetorical material to construct a biography about the legal subject to be sanctioned, and the complex quantitative guidelines system becomes incorporated into narrative form to know, assess, and judge legal subjects \cite{101111lsi12334}. The sociology of quantification treats statistics as cultural objects formed through social practices, revealing how figures, indicators, and rates become "public artifacts" that interpret and shape social realities, reflecting certain interests and biases and contributing to power dynamics and social justice issues \cite{camargo2021,paiva2021}.

Quantitative methods in law can reproduce and amplify existing social biases and inequalities when algorithms and risk assessment tools are trained on historical data that reflects past discriminatory practices, perpetuating these biases in future decision-making and disproportionately impacting marginalized groups \cite{10.1590/dados.2022.65.3.267,10.3390/fi9040068}. The use of opaque algorithms that predict case outcomes can make it difficult to challenge any injustice arising from flawed quantification, and the focus on individual-level data in quantitative legal analysis can obscure the structural and systemic factors that contribute to social inequalities, leading to a narrow understanding of legal issues that fails to address the root causes of injustice \cite{10.1590/dados.2022.65.3.267,10.3390/fi9040068}. This critical lens helps unmask biases that can become embedded in algorithms and data analysis techniques, potentially leading to unjust outcomes \cite{10.1590/dados.2022.65.3.267,10.3390/fi9040068}.

However, when used critically and reflexively, quantitative methods in law can help identify and correct biases in legal decision-making by analyzing patterns and disparities in legal outcomes, enabling researchers and practitioners to develop more equitable and just legal practices \cite{10.1590/dados.2022.65.3.267,10.3390/fi9040068}. Implementing transparent and accountable practices in the use of quantitative methods in law, including making the assumptions, data, and algorithms used in legal decision-making open to public scrutiny and debate, is essential to ensure that the quantification process is transparent and participatory, allowing for the inclusion of diverse perspectives and experiences \cite{10.1590/dados.2022.65.3.267,10.3390/fi9040068,salais2016}.

The transformative power of quantification highlights the importance of critically examining the social, political, and ethical implications of using quantitative methods in law \cite{10.5040/9781350220645,10.1080/07329113.2015.1046739}. Statistical models trained on large datasets of legal cases can be used to forecast potential legal outcomes, aiding lawyers in developing data-driven legal strategies and policymakers in crafting evidence-based legislation, but it is essential to recognize the probabilistic nature of such predictions and the need to consider the inherent complexities and uncertainties of legal systems \cite{10.5040/9781350220645,10.1080/07329113.2015.1046739}. Jurisprudence and science have fundamentally incompatible methodologies and objectives, with jurisprudence being normative and science empirical, and the study involving rats highlights the importance of acknowledging sampling errors and the limitations of representativeness in experimental studies \cite{loevinger1949,nunes2018}.

Quantification in law refers to the process of assigning numerical values to legal concepts, which can help make legal decisions more objective and consistent, particularly in areas such as damages, sentencing, and risk assessment \cite{damages_quantification}. Quantification also plays a role in regulatory law, where agencies often use cost-benefit analysis to evaluate the potential impact of proposed regulations by quantifying the expected benefits and costs in monetary terms, helping policymakers make informed decisions about whether to implement the regulation \cite{cost_benefit_analysis,environmental_regulation}. However, the process of quantification in law is fraught with challenges, such as the difficulty of assigning numerical values to complex and often subjective concepts, which can lead to debates about the accuracy and fairness of the quantification process \cite{quantification_challenges}. There is also a risk that an over-reliance on quantification can lead to a reductionist approach to legal decision-making, where important qualitative factors are overlooked \cite{reductionist_approach}.

Despite these challenges, quantification remains a valuable tool in the legal system, helping to bring a degree of objectivity and consistency to legal decision-making \cite{quantification_value}. Leveraging technological advancements, such as legal databases and data-mining programs, is crucial for objective analysis, and combining legal, statistical, and computational disciplines enables the design of research tests and replicable assessments, applying statistical models to legal issues to uncover biases and trends, ultimately guiding fairer legal reforms \cite{nunes2018,massuanganhe2016,maia2019}. Involving diverse stakeholders in the development and governance of quantitative legal tools is essential, engaging with communities affected by these tools to ensure that their perspectives and experiences are taken into account \cite{10.1007/s11186-021-09453-1,10.5040/9781350220645}. This participatory approach is emphasized by the sociology of quantification, which argues for the inclusion of diverse perspectives to ensure that the tools developed are fair and just \cite{10.1590/dados.2022.65.3.267,10.1057/s41599-020-0396-5}.

In summary, while quantification in the legal system offers significant benefits in terms of objectivity and consistency, it also presents challenges that must be carefully managed. By involving diverse stakeholders and leveraging technological advancements, the legal system can use quantification to enhance fairness and effectiveness, ultimately leading to more just outcomes.

\subsection{Quantification, Justice, and Inequality}

The sociology of quantification emphasizes the importance of public discourse and democratic deliberation throughout the quantification process, promoting a multifaceted perspective that counters the inherent biases of quantification and invites interdisciplinary collaboration \cite{10.1590/dados.2022.65.3.267,10.1080/07329113.2015.1046739}. Inconsistencies within legal judgments and enabling a notably more equitable legal structure \cite{10.1007/s11186-021-09453-1,10.5040/9781350220645}. The unethical use of algorithms and the unintended consequences of their application are significant concerns \cite{10.1057/s41599-020-0396-5,10.1057/s41599-020-00557}. The application of quantification in law presents several risks, including the potential commodification of justice and the exclusion of non-legal arguments, which can impair holistic legal reasoning \cite{nunes2018, ribeiro1998}. Despite these risks, jurimetrics can reveal biases, improve the management of legal processes, and provide empirical evidence for legal reforms, supporting a more transparent, systematic, and data-driven approach to law \cite{ribeiro2021, nunes2018, silva2023}. The use of jurimetrics, which involves the application of quantitative methods to legal decision-making, has the potential to both illuminate and obscure legal realities, potentially reinforcing existing power imbalances \cite{10.1057/s41599-020-00557-0,10.1590/dados.2022.65.3.267}. Despite its potential, jurimetrics faces several challenges. One significant issue is the increasing reliance on quantitative methods in legal decision-making, which raises ethical and legal concerns \cite{jurimetricschallenges}. The lack of standardization in the collection and reporting of legal data has further limited the potential impact of jurimetrics \cite{jurimetricschallenges}. Additionally, the ethical implications of using predictive analytics in legal decision-making, such as the potential for bias and the risk of over-reliance on quantitative data at the expense of qualitative analysis, present significant challenges.

\section{Jurimetrics as an Object of the Sociology of Quantification}
    % Fundamentação teórica
% ------------------------------------------------------------------------------
% Trabalhos Relacionados
% ------------------------------------------------------------------------------

\chapter{Trabalhos Relacionados}
\label{chap_trabalhos_relacionados}

Cada capítulo deve conter uma pequena introdução (tipicamente, um ou dois parágrafos), que deve deixar claro o objetivo e o que será discutido no capítulo, bem como a organização do mesmo.
   % Trabalhos relacionados
% ------------------------------------------------------------------------------
% Metodologia
% ------------------------------------------------------------------------------

\chapter{Metodologia}
\label{chap_metodologia}

As ferramentas que serão utilizadas para atingir os objetivos deste estudo baseiam-se em técnicas de revisão bibliográfica dos principais livros, bem como de artigos veiculados em periódicos científicos, sobre quantificação, sociologia da quantificação, análise econômica do Direito, teoria das origens jurídicas, análise jurídica da política econômica, \emph{Law \& Finance} e jurimetria, enfatizando as publicações apresentadas nos últimos quinze anos.

Essas técnicas são (1) a utilização de serviços de indexação de citações, que oferecem índices de citações entre publicações e mecanismos para estabelecer quais documentos citam quais outros documentos e (2) o método de seguir trilhas bibliográficas que, conjuntamente, permitirão a construção de uma rica rede de fontes que sustentarão este estudo por meio de sua análise através de abordagens dialéticas e hermenêuticas \cite{booth2003craft}.

\section{Coleta e tratamento de dados}
\label{sec_coleta_e_tratamento_de_dados}

Inserir seu texto aqui...

\section{Análise estatística}
\label{sec_analise_estatistica}

Inserir seu texto aqui...
              % Metodologia
% ------------------------------------------------------------------------------
% Resultados
% ------------------------------------------------------------------------------

\chapter{Resultados}

Cada capítulo deve conter uma pequena introdução (tipicamente, um ou dois parágrafos), que deve deixar claro o objetivo e o que será discutido no capítulo, bem como a organização do mesmo.

\section{Discussão}
\label{sec_discussao_resultados}

Inserir seu texto aqui...
               % Resultados
% ------------------------------------------------------------------------------
% Conclusão
% ------------------------------------------------------------------------------

\chapter{Conclusão}
\label{chap_conclusao}

Procure fazer uma análise crítica de seu trabalho, destacando os principais resultados e as contribuições deste trabalho para a área de pesquisa.

\section{Trabalhos futuros}
\label{sec_trabalhos_futuros}

Também deve indicar, se possível e/ou conveniente, como este trabalho pode ser estendido ou aprimorado.

% ------------------------------------------------------------------------------
% Observação: A norma ABNT estabelece que, em qualquer categoria de trabalho
% acadêmico monográfico deve haver um capítulo de conclusão
% ------------------------------------------------------------------------------
                % Conclusão

\postextual % Define o estilo de página para os elementos pós-textuais

\imprimirreferencias{referencias.bib}                   % Referências
% ------------------------------------------------------------------------------
% Apêndices
% ------------------------------------------------------------------------------

\begin{apendicesenv}
    \partapendices

    % --------------------------------------------------------------------------
    % Primeiro apêndice
    % --------------------------------------------------------------------------

\end{apendicesenv}
            % Apêndices
% ------------------------------------------------------------------------------
% Anexos
% ------------------------------------------------------------------------------

\begin{anexosenv}
    \partanexos

    % --------------------------------------------------------------------------
    % Primeiro anexo
    % --------------------------------------------------------------------------


\end{anexosenv}
               % Anexos
\printindex                                             % Índice remissivo

\end{document}

% ------------------------------------------------------------------------------
% Fundamentação Teórica
% ------------------------------------------------------------------------------

\chapter{Fundamentação Teórica}
\label{chap_fundamentacao_teorica}

\section{Quantificação como objeto sociológico}

Conforme mencionado por Didier, a quantificação é uma ferramenta de poder e governança que produz conhecimento sobre o mundo, afetando diversos aspectos da vida social como as relações interpessoais, a gestão organizacional, a avaliação de condutas, a consolidação dos estados modernos e a evolução dos mercados financeiros. Tornou-se uma ferramenta relevante para o desenvolvimento de outras áreas do conhecimento intimamente ligadas às realidades sociais, como a sociologia e o direito.

\subsection{Quantificação segundo Alain Desrosières}

A quantificação, processo que envolve o ato de contar e medir, desempenha um papel significativo na sociedade. É uma ferramenta fundamental na aquisição e gestão do conhecimento, permitindo-nos dar sentido ao mundo que nos rodeia. Ao converter dados qualitativos em forma numérica, a quantificação nos permite categorizar, classificar e analisar vários fenômenos. Esse processo está profundamente enraizado em nossas estruturas sociais, influenciando nossa compreensão e interpretação da realidade \cite{boltanski2011critique}.

Um exemplo do uso da quantificação na classificação social pode ser encontrado no trabalho de Alain Desrosières, um renomado estatístico e sociólogo francês. No século XX, Desrosières empreendeu um estudo empírico e reflexivo das categorias socioprofissionais na França a partir de uma perspectiva estatística. Seu trabalho apresenta alguns dos papéis da quantificação na formação de estruturas e percepções sociais.

Desrosières descobriu que durante o processo de coleta de dados populacionais, os entrevistados relataram uma ampla gama de profissões e a tarefa do estatístico era atribuir a essas profissões um certo número de categorias que representassem grupos socialmente homogêneos \cite{desrosieres2016quantification}. Essa categorização partiu do pressuposto de certa afinidade entre pessoas de uma mesma categoria e a nomenclatura dessas categorias, segundo Desrosières, refletia o estado das lutas de classes e a configuração atual das relações de poder nas quais o estatístico estava imerso \cite{camargo2009sociologia}.

Ademais, as classificações produzidas pelo estatístico conformaram-se às expectativas formais dos grupos dominantes, produzindo categorias estáveis que, para esses grupos, representavam o mundo de forma realista e coerente. No entanto, essa representação não era neutra ou objetiva. Esse resultado já estava predeterminado pelas expectativas definidas pelos detentores do poder local. Nesse contexto, o papel do estatístico no processo de quantificação não era apenas técnico, mas também político, destacando a lógica hegemônica da quantificação, em que números e categorias são usados para reforçar estruturas de poder e normas sociais existentes \cite{salais2016quantification}.

O trabalho de Desrosières desafia o entendimento tradicional das estatísticas como objetivas e imparciais ao destacar as dimensões sociais e políticas das práticas estatísticas, argumentando que esses processos estão profundamente enraizados em contextos sociais, históricos e epistemológicos. Para ele a compreensão das categorias na estatística reside na constante oscilação entre dois pólos opostos, mas complementares: o convencional e o real. A dimensão convencional refere-se aos aspectos socialmente construídos das categorias, enquanto a dimensão real diz respeito aos seus aspectos tangíveis, mensuráveis. Essa perspectiva ressalta a relação dialética entre a construção social das categorias e sua validação empírica, destacando a complexa interação entre quantificação, sociedade e poder. Esse trabalho de Desrosières apresenta um ponto de vista mais crítico por meio do qual é possível a compreensão do papel e do impacto da quantificação na sociedade. Além disso, essa obra ressalta a importância de reconhecermos as dimensões sociais e políticas das práticas estatísticas a fim de melhor compreendermos o mundo através de análises mais críticas aos números e categorias que nos são dados.

\subsection{Desenvolvimento da quantificação na sociedade}

As origens da quantificação remontam à Idade Média, quando o poder do príncipe baseava-se na propriedade de territórios e o domínio dessa ferramenta para administração dos recursos do príncipe era mister para a manutenção de seu poderio. Desde então a quantificação vem desenvolvendo-se na história da humanidade por meio da sua relação com a categorização e a classificação, pois quantificar aos poucos foi deixando de ser somente medir e expressar algo em números, a exemplo da sua aplicação na sociologia, permitindo mensurar, comparar e analisar vários aspectos da sociedade \cite{camargo2021social}. Ao contrário da aritmética, na quantificação, seu uso envolve não apenas números, mas também pessoas, respostas a questionários ou qualquer outra entidade no mundo. Isso destaca a contribuição das práticas quantitativas para a produção de conhecimento. Um exemplo do seu desenvolvimento histórico é o slogan do século XIX \emph{Um homem, um voto}. Esse slogan, que era um apelo à igualdade, excluía as mulheres pelas regras da convenção da época sobre quem poderia compor a categoria eleitora \cite{berman2018sociology}. Esse exemplo ilustra que a quantificação não é apenas um exercício mecânico ou matemático, mas uma complexa interação de convenções, classificações e medições. O processo de tradução de fenômenos sociais complexos em dados numéricos envolve os dois supracitados conceitos-chave: classificação e categorização. A categorização é a atribuição de uma entidade a uma categoria, enquanto a classificação envolve a atribuição de uma entidade a uma categoria e a avaliação da entidade relacionando-a a uma classe \cite{mennicken2019s}. Compreender esses significados e a diferenciação entre classificação e categorização é crucial para entender o conceito de quantificação no campo da sociologia. Ademais, a própria sociologia pode ser usada para analisar os processos centrais de quantificação e seus pré-requisitos socioepistemológicos, todavia o primeiro contato da quantificação com a sociologia pautou-se no desempenho de papéis significativos na compreensão dos fenômenos sociais, na avaliação de políticas e intervenções e na análise da organização social. Eventualmente o papel da quantificação na sociedade foi evoluindo bem como seu uso para servir como ferramenta de poder e governança por estar profundamente enraizada nos processos sociais e políticos e por moldar nossas percepções da realidade e influenciar a forma como percebemos e entendemos o mundo \cite{camargo2021social}.

Então, a quantificação na sociologia é um processo complexo que envolve a interação de convenções, classificações e medições e é uma ferramenta que tem sido usada historicamente para exercer poder e influenciar a organização da sociedade. Dessarte, seu estudo é essencial para a compreensão dos fenômenos sociais e para o desenvolvimento de políticas e intervenções efetivas na sociedade.

\subsection{O papel da quantificação nos assuntos de Estado}

O ato de quantificar tornou-se um expediente significativo para organizar a sociedade e administrar os Estados \cite{desrosieres1998politics}. A ascensão dos números em assuntos de Estado pode ser rastreada até o século 18, marcando uma mudança no modelo tradicional de gestão familiar. Este modelo foi substituído pela noção de população como um recurso fundamental do poder do Estado. Desde então, os números tornaram-se importantes mediadores das tecnologias contemporâneas de governo, apoiando intervenções que visam o corpo social. Essas intervenções se concentram em processos biológicos, como nascimentos, mortes, estado de saúde, expectativa de vida e longevidade. Essa racionalização do poder como prática de governo, conhecida como governamentalidade, envolve um conjunto complexo de instituições, procedimentos, análises, cálculos e táticas que visam a população como foco principal de conhecimento e controle \cite{diaz2016convention}. A quantificação, nesse contexto, serve de prova numérica para descrever a realidade, ganhando assim legitimidade científica e social para sua ação sobre essa realidade como ferramenta do governo. Essa dupla dimensão das estatísticas como instrumento de prova e ferramenta de governo ressalta sua importância nos assuntos de Estado \cite{bruno2014statactivism}.

Ademais, elas moldam as políticas públicas e refletem as expectativas sociais de vários grupos que aspiram ser legitimados como membros da sociedade local a fim de tornarem-se visíveis para tais políticas, como as de proteção social. A quantificação de vários aspectos da sociedade também permite o monitoramento e a avaliação das ações governamentais por meio de indicadores econômicos, indicadores sociais, dados demográficos etc. Esses números servem como importantes mediadores de tecnologias contemporâneas de governo, fornecendo uma base para a tomada de decisões e implementação de políticas. No entanto, o uso de números em assuntos de Estado tem seus desafios e limitações. A confiança na quantificação pode levar a uma simplificação excessiva de realidades sociais complexas, e a legitimidade dos números pode ser contestada diante de dados ou interpretações conflitantes \cite{camargo2021social}. Além disso, o uso de números pode ser manipulado para atender a interesses particulares, levantando questões sobre a transparência e a responsabilidade das ações governamentais. 

Esse aumento de números em assuntos de Estado e seu uso como uma ferramenta de governança têm profundas implicações para a sociedade. Embora a quantificação forneça um meio de organizar a sociedade e administrar os Estados, ela também levanta questões importantes sobre a legitimidade, transparência e responsabilidade das ações governamentais \cite{berman2018sociology}. Como tal, uma compreensão crítica do papel da quantificação nos assuntos do Estado é essencial para uma tomada de decisão informada e uma governação eficaz.

\subsection{Estatísticas, sua dupla dimensão e a realidade}

Uma das roupagens da quantificação é a estatística, campo que possui diversas dimensões, como a cognitiva. A dimensão cognitiva da estatística é um conceito complexo que se relaciona com o processo de quantificar, a saber, o estabelecimento de categorias e classificações e a aplicação do princípio da equivalência. Tal dimensão dos sistemas estatísticos não se limita ao domínio estatal ou das ciências, mas permeia todos os aspectos da vida social, influenciando nossa compreensão e interpretação do mundo que nos cerca.

O processo de quantificação envolve uma série de etapas, incluindo o estabelecimento de classificações, categorizações e de convenções de equivalência, que servem de base para a estrutura cognitiva dos sistemas estatísticos. O princípio da equivalência, conforme conceituado por Desrosières, serve como a lógica subjacente a esses processos. Essa conceituação do princípio da equivalência destaca a alhures dupla dimensão das categorias, que oscilam entre o convencional e o real, ressaltando a complexidade inerente e o dinamismo dos sistemas estatísticos.

Seguindo no processo de quantificação, as medições de entidades devem ser precedidas por convenções sobre essas mesmas entidades. Essas convenções, baseadas no princípio da equivalência, servem como a mencionada lógica implícita sobre a qual se baseiam as categorias e classificações, demonstrando como a quantificação é criada e, por sua vez, cria o mundo.

Dessarte a quantificação não apenas reflete a realidade; ela a constrói ativamente. Isso é feito por meio da criação de regras convencionais, que atribuem entidades, consideradas equivalentes entre si, a categorias de padronização. Essas regras não são simplesmente espelhos da realidade, mas são construções ativas que constituem e transformam a realidade.

No entanto, é importante notar que os princípios de classificação lógica só geram valor para um determinado grupo de categorias. Essa limitação ressalta a necessidade de uma compreensão diferenciada das dimensões cognitivas da estatística, a saber, entender como o processo de quantificação e a aplicação do princípio da equivalência são influenciados por uma variedade de fatores, incluindo contextos sociais, culturais e políticos.

\subsection{Espaços de equivalências cognitivas}

Outro conceito importante explanado por Desrosières é o relacionado à estruturação de espaços de equivalências cognitivas, onde determina-se o alcance político e geográfico de categorias e classificações por meio da influência de delimitações históricas e de lutas de classes. Esses espaços, utilizados como pontos de referência para debates e discussões, têm sido fundamentais para o uso - anteriormente mencionado - da quantificação pelo Estado. Tal uso como ferramenta de governo por instituições ocorre através da fixação de equivalências - por meio de consensos - nesses espaços, materializando, assim, categorias e classificações. Essas instituições, de natureza abstrata, criam tais consensos por meio do já mencionado processo de fixação de categorias que estruturam ordenadamente espaços de equivalências cognitivas. Assim, a utilização da quantificação pelo Estado como ferramenta de governo ocorre por meio do estabelecimento de equivalências, sob a lógica de princípios de equivalência, nesses espaços cognitivos. A ênfase no termo equivalência nesses conceitos se deve à especificidade da linguagem de quantificação, que permite comparações, transferências, agregações e manipulações padronizadas por meio de cálculos matemáticos. As equivalências cognitivas visam melhorar a governança por meio da produção de informações que auxiliam na descrição, gestão e transformação de grupos sociais. Alcança-se isso através da criação de instrumentos de ação pública, como indicadores que compilam dados populacionais contados e classificados, fornecendo informações cruciais para o exercício do poder.

A política, a administração e a vida pública estão cada vez mais interligadas com a produção e divulgação de dados estatísticos. Organizações internacionais, organizações não-governamentais e atores privados desempenham papéis significativos nesse processo, muitas vezes com implicações políticas significativas \cite{desrosieres2008argument}. As regras para a produção de indicadores, rankings e metas passaram a fazer parte da lógica política, servindo de base para debates e decisões políticas. Essas avaliações quantificadas do desempenho público são usadas para legitimar ou desafiar a autoridade política. A chamada cultura de resultados atesta o papel fundamental que os números desempenham não apenas na medição de realidades objetivas, mas na construção de identidades sociais e das próprias realidades \cite{berman2018sociology}.

\subsection{Engajamento crítico à quantificação}

Classificar e categorizar não são atos neutros, mas estão profundamente enraizados na política, evidenciando a dinâmica de poder inerente que muitas vezes é negligenciada no processo de quantificação. As escolhas feitas nesse processo não são arbitrárias, mas são influenciadas por uma série de fatores, incluindo normas sociais, interesses políticos e contextos históricos. Esses fatores, por sua vez, moldam a maneira como entendemos e interagimos com o mundo ao nosso redor. Números e dados estatísticos não são apenas ferramentas neutras para entender a realidade; são também instrumentos que sustentam argumentos políticos e legitimam certas ordens sociais \cite{chun2021logic}. A utilização de números e dados estatísticos na política é uma prova do seu poder de formação da opinião pública e da política. Eles são usados para justificar decisões, alocar recursos e estabelecer prioridades, refletindo e reforçando as estruturas de poder existentes. As classificações produzidas pelo processo de quantificação não são meramente descritivas; elas refletem as lutas de classes dentro da sociedade. Elas são o produto da negociação e contestação entre diferentes atores sociais, cada um com seus próprios interesses e perspectivas \cite{berman2018sociology}. Isso destaca a construção social das estatísticas, que é um processo que depende de eventos históricos e forças sociais. As estatísticas não refletem simplesmente a realidade; eles contribuem para criá-la. Este é um ponto crucial para compreender a dimensão política da quantificação. 

Os números que usamos para descrever o mundo não representam apenas uma realidade objetiva; eles também moldam nossa percepção dessa realidade. Essa dupla dimensão revela a complexa interação entre estatísticas e quantificações como construtos sociais e o mundo que ambas produzem e são produzidas por \cite{desrosieres1998politics}. As estatísticas permitem a geração de conhecimento. Elas fornecem uma estrutura para entender e interpretar o mundo e, assim, permitem o exercício do poder. O conhecimento gerado pela estatística não é neutro; é moldado por essas relações de poder que fundamentam o processo de quantificação.

Todavia, esse conhecimento, por sua vez, pode ser usado não somente para reforçar as estruturas de poder existentes, mas também para desafiá-las. As relações de poder engendradas pela quantificação não são apenas um subproduto do processo; eles são parte integrante dele, pois o ato de quantificar é um ato de poder que envolve definir o que é importante, o que é mensurável e o que vale a pena conhecer. Esse processo é dinâmico e pode ser manipulado - conforme o engajamento crítico de cada ator - pelas ordens sociais e estruturas de poder dentro das quais ocorre \cite{espeland2008sociology}.

\subsection{Categorias como construtos sociais}

O Estado, embora seja um ator significativo, não é o único envolvido na produção e utilização de estatísticas. A geração de dados está se tornando cada vez mais um assunto privatizado, e as convenções que sustentam esse processo geralmente são obscuras e opacas. Essa obscuridade das convenções não é mera coincidência, mas um reflexo das complexas dimensões cognitivas da quantificação \cite{desrosieres1988categories}. A dimensão cognitiva - já explanada - da quantificação diz respeito a como a quantificação molda nossa compreensão do mundo. E isso atinge significativamente as estruturas cognitivas dos atores sociais quando tais processos acabam sendo naturalizados, em que os produtos dos procedimentos de quantificação são aceitos como inevitáveis.

Devido a isso é essencial entender esses processos e suas implicações para apreciar com criticidade o papel da quantificação na produção de conhecimento e na formação de nosso mundo. Novamente, os processos de quantificação não oferecem apenas um reflexo do mundo; transformam-no e reconfiguram-no de outra forma \cite{camargo2021estudos}. Esta transformação não é um simples detalhe técnico mas tem implicações históricas, políticas e sociológicas. Evidencia-se isso nos fenômenos sociais que podem ser pensados, expressos, definidos e quantificados de múltiplas formas, cada uma com seu próprio conjunto de divergências e pontos de vista por serem processos sociais que envolvem negociação, contestação e luta de poder entre diversos atores. Essas características contestam a noção da estatística como inevitável ou natural e corroboram sua capacidade de alterar as relações de poder afetando como os recursos, status, conhecimento e oportunidades são distribuídos \cite{didier2021estatativismo}.

Desafiar essa naturalização das estatísticas se dá por meio da compreensão de que elas são construtos sociais, a saber, que não são dadas, mas são construídas socialmente para refletir os interesses e o poder daqueles que as constroem. A exemplo, a construção de categorias como raça, etnia e classe social, e a subsequente quantificação dessas categorias, podem moldar os entendimentos sociais e influir nas relações de poder existentes.

Ademais, importante compreender o poder das instituições na definição das coisas e, portanto, na construção da realidade, pois elas desenvolvem estatísticas públicas com a pretensão subjacente de consolidar sua própria existência. Visam fortalecer a dependência dos indivíduos em relação a elas, na ideia de que somente elas seriam capazes de sanar conflitos advindos de divergências entre pontos de vista de atores sociais. Essa dependência das instituições é uma prova de seu papel na definição da realidade, no estabelecimento da ordem e na naturalização dos produtos dos procedimentos de quantificação.

\subsection{Mundo, realidade e quantificação}

O mundo, como o entendemos, representa o fluxo de eventos, experiências e contingências da existência humana. É uma entidade complexa e em constante mudança que talvez nunca possamos compreender ou controlar totalmente devido à sua imprevisibilidade inerente e às limitações de nossa capacidade humana \cite{mottaresisting}. Este conceito de mundo encapsula a totalidade das experiências humanas, tanto tangíveis quanto intangíveis, que estão constantemente mudando. A realidade, por outro lado, só pode capturar uma fração da totalidade do mundo. Isso se deve a limitações intrínsecas, como a capacidade cognitiva humana bem como vieses e incompletude dos dados disponíveis. O processo de quantificação, embora útil para dar sentido a certos aspectos do mundo, muitas vezes é insuficiente para captar toda a complexidade e riqueza das experiências vividas \cite{hirata2021operaccoes}. Essa discrepância entre quantificação, realidade e mundo é uma área sensível, pois pode levar a distorções e deturpações do mundo como o conhecemos. Na obra de \citeonline{boltanski2011critique}, a realidade representa os arranjos, classificações e avaliações institucionalizadas que procuram dar sentido a este mundo. Esses arranjos estão intrinsecamente ligados às estruturas de poder social e político, que muitas vezes ditam as regras e normas que regem nossa compreensão da realidade. Porém, essas instituições que produzem classificações não são infalíveis \cite{camargo2022estado}. Elas estão sujeitas a dar ensejo a discrepâncias e preconceitos, tornando-as vulneráveis a críticas e questionamentos. Os choques e contradições potenciais dentro desse processo de construção da realidade oportunizam a crítica. É por meio dessa tensão que a crítica surge como uma ferramenta essencial para revelar e desafiar as estruturas de dominação \cite{boltanski2011critique}.

A evidenciação dessas discrepâncias permite o questionamento da autoridade dessas instituições e desafiar a reivindicação dessas regras. A exemplo, considere o caso das estatísticas públicas de distribuição de renda de uma instituição estadual \cite{susen2014spirit}. Essas estatísticas, embora retratem uma realidade de distribuição equitativa de riqueza, podem não estar alinhadas com as experiências vividas pelos cidadãos. A discrepância entre estas estatísticas públicas e as experiências vividas pelos cidadãos pode revelar desigualdades sociais gritantes, desafiando a autoridade da instituição e a sua representação da realidade. A quantificação muitas vezes desempenha um papel proeminente na produção da autoridade dos fatos. No entanto, é fundamental enfrentar essa autoridade de forma crítica e reflexiva \cite{boltanski2011critique}. Devemos questionar como as estatísticas e os dados são manipulados a fim de verificar eventuais deturpações e usos para enganar e iludir. Essa reflexão crítica nos permite questionar a seletividade desses consensos institucionais, que muitas vezes resultam da influência de determinações históricas pautadas pelas lutas de classes.

\subsection{Quantificação e dominação}

A reação das instituições às críticas sufocando-as ou resistindo a elas é um fenômeno característico das tentativas da classe dominante em manter o controle. Essa resistência à crítica tem o potencial de levar a um tratamento injusto nos processos de quantificação, o que pode minar ainda mais a legitimidade dessas instituições \cite{keslassy2014isabelle}. Essas instituições muitas vezes afirmam, como já mencionado, ser o único e exclusivo padrão da verdade, condição que simboliza certa violência aos indivíduos. Esse posicionamento induzido os obriga a se alinhar no espaço social de acordo com essas regras, muitas vezes distorcidas em favor da classe dominante. Esta, em virtude de seu controle sobre essas instituições, é capaz de limitar as faculdades reflexivas e críticas do restante da população tornando mais fácil para a classe dominante a exploração dos oprimidos para obtenção de lucro \cite{starr1992social}. Essa exploração é muitas vezes mascarada pelo uso da quantificação, que pode sustentar distinções sociais opressivas e exacerbar as desigualdades sociais.

O uso de estatísticas na formulação de políticas pode ter impactos significativos sobre as desigualdades sociais, especialmente em regiões como a América Latina, onde as disparidades sociais são gritantes. Um exemplo histórico disso é a quantificação dos negros como 3/5 de uma pessoa nos censos americanos do século XIX, resultado de um consenso alcançado durante a Convenção Constitucional dos Estados Unidos de 1787.

Reforçando o que já foi explanado alhures, as estatísticas informam nosso pensamento e tomada de decisões cotidianas, lapidando nossas interações e relacionamentos sociais. Elas influenciam nossos valores e normas culturais, moldam nossas percepções e expectativas, nossas identidades e subjetividades e nosso senso do que é possível e desejável \cite{hacking1990taming}. Na sociedade capitalista contemporânea, as estatísticas desempenham um papel mister como instrumento governamental para reger a população ao ajustar a distribuição espacial dos indivíduos à acumulação de capital, ao crescimento de grupos e à distribuição diferencial de lucros. Porém, como também já mencionado, os efeitos da quantificação não se restringem a determinados grupos sociais, mas se espalham por toda a sociedade, influenciando as formas como percebemos e interagimos com o mundo e possibilitando a apropriação da quantificação como ferramenta para os movimentos sociais, permitindo-lhes contestar o domínio de instituições poderosas e questionar as realidades que essas instituições promovem \cite{camargo2021estudos}.

\subsection{Estatativismo}

Estatativismo, um termo cunhado por \citeonline{didier2021estatativismo}, encapsula o uso estratégico da quantificação pelos movimentos sociais para desafiar o domínio dessas instituições poderosas e questionar as realidades que promovem. Este conceito baseia-se na crença de que números e estatísticas não são neutros, mas sim ferramentas que podem ser usadas para defender ou desafiar as estruturas de poder existentes. 

O estatativismo é visto como uma forma de resistência contra a opressão e a injustiça social, tendo como objetivo último a emancipação política. É uma crítica ao status quo, um apelo à desconstrução de narrativas que muitas vezes são tidas como certas, e um apelo à criação de novas equivalências que melhor reflitam as complexidades do nosso mundo. Os estativistas aproveitam o poder da quantificação para criticar e desconstruir a realidade criada por meio dos números, permitindo assim o estabelecimento de novas equivalências. Eles usam a estatística não apenas como uma ferramenta de descrição, mas como uma arma para denunciar realidades e formalizar críticas para influenciar e reformar governos. Eles são participantes ativos que procuram intervir na realidade de forma a promover a justiça e a igualdade. Em sua essência, o estativismo é uma resposta à lacuna inevitável entre o mundo como ele é e a realidade institucionalizada que tentamos impor a ele. Enquanto o ato de quantificar permanece fundamental na organização das sociedades e na administração dos Estados, o estativismo propõe aproveitar esse poder em toda a sociedade, especialmente entre os oprimidos, e não deixá-lo exclusivamente nas mãos de instituições poderosas. É um chamado para democratizar o poder de quantificação, para torná-lo uma ferramenta que pode ser usada por todos para desafiar e remodelar o mundo.

O estatativismo prospera nas discrepâncias que existem nas instituições que produzem classificações. É nessas discrepâncias que os estativistas encontram espaço para questionar, desafiar e, finalmente, mudar as narrativas dominantes. Eles usam a quantificação como uma ferramenta de luta e resistência contra a opressão e a injustiça social, estabelecendo coletivos plurais que buscam contestar a realidade criada através dos números.

Reforçando de outro forma tal conceito que é complexo, o estatativismo se beneficia quando as temporalidades das classificações no espaço público não correspondem mais às temporalidades sociais. Esse desalinhamento cria oportunidades para os estativistas desafiarem as narrativas dominantes e proporem alternativas que reflitam melhor as realidades do mundo \cite{lara2019acesso}. Eles fazem uso de estatísticas como uma forma de ativismo político, por exemplo, no caso da comunidade LGBTQ em que a publicação do relatório Kinsey em 1948 revelou que a proporção de homens que tiveram relações exclusivamente homossexuais ao longo da vida era muito mais alta do que se pensava anteriormente. Essa revelação desafiou a narrativa dominante sobre a homossexualidade e abriu caminho para uma maior aceitação e direitos para a comunidade LGBTQ.

Então, o estatativismo revela a falibilidade, o domínio e a opressão inerentes às realidades institucionalizadas. O poder da quantificação, revelado por meio do estativismo, oferece uma ótica crítica para a compreensão dos assuntos do Estado e das práticas institucionais. Dessarte, o Estatativismo não é apenas sobre números; trata-se de usar esses números para criar um mundo mais justo e igualitário.

\subsection{Quantificação como objeto sociológico}

A discussão acerca do Estatativismo enseja uma abordagem crítica abrangente das dimensões políticas e cognitivas das estatísticas e quantificações. Essa crítica desafia a neutralidade e a objetividade frequentemente assumidas das estatísticas e quantificações e expõe seus preconceitos sociais, políticos e culturais inerentes. É essencial salientar que esses vieses não são meramente acidentais, mas muitas vezes estão embutidos nos próprios métodos e suposições que sustentam a coleta, análise e interpretação dos dados. A validade e a confiabilidade das estatísticas e dos dados também devem ser questionadas nessa crítica \cite{lowrey2019data}. Os métodos usados para coletar dados, as suposições feitas no processo e as interpretações extraídas dos dados devem ser todas examinadas. Isso não é para minar o valor das estatísticas e quantificações, mas sim para garantir seu bom uso e evitar abusos por meio de manipulações ou deturpações para servir a interesses ou agendas particulares \cite{espeland2008sociology}. Compreender essa abordagem crítica é fundamental para uma utilização responsável e ética dessa ferramenta tão poderosa.

A estatística, quando usada com responsabilidade, pode ajudar a desmistificar falsas impressões sobre a realidade. No entanto, elas também podem servir para reforçar equívocos e preconceitos existentes, especialmente quando são usadas sem uma compreensão crítica de suas limitações e potenciais armadilhas. O uso de números na esfera estatal, por exemplo, tem seu próprio conjunto de desafios e limitações \cite{sareen2020ethics}. Os críticos argumentam que uma confiança excessiva na quantificação pode levar a uma visão estreita e reducionista da realidade, resultando em uma simplificação excessiva de questões sociais complexas e na negligência de fatores importantes devido à confiança em números, a exemplo do uso de testes padronizados na educação, já que o complexo processo de aprendizagem é muitas vezes reduzido a uma única pontuação numérica, o que pode levar a uma simplificação excessiva do processo de aprendizagem e à exclusão de aspectos qualitativos importantes \cite{laevers1994innovative}.

Da mesma forma, a classificação e a aferição de crimes refletem certas interpretações da lei e da ordem e moldam nossa compreensão do crime e da criminalidade de maneiras que podem não captar totalmente a complexidade desses fenômenos. Há uma crítica crescente contra a autoridade dos fatos, particularmente em relação a como estatísticas e dados são usados para justificar e legitimar certas políticas e práticas, e para marginalizar e excluir outras. Essa crítica visa capacitar grupos sociais para compreender as estatísticas públicas de forma crítica e reflexiva como construções sociais ligadas ao poder, rompendo com os pressupostos do senso comum de que tais números são fatos objetivos \cite{sellar2018feel}.

Para tanto, a sociologia da quantificação surge como espaço de crítica e resistência ao envolver a investigação da manipulação e deturpação de estatísticas e dados, e a crescente privatização da coleta e análise de dados \cite{gillborn2018quantcrit}. Esse campo de estudo também destaca as limitações da quantificação, particularmente em relação ao potencial de uso indevido e abuso de dados.

\subsection{A sociologia da quantificação}

A sociologia da quantificação, um campo de estudo em rápida expansão, preocupa-se principalmente com as implicações sociais da quantificação, do processo de transformação de fenômenos sociais complexos em dados numéricos \cite{berman2018sociology}. Esse campo examina criticamente a produção e comunicação de números, incluindo gráficos como representações visuais de dados numéricos, em relação ao poder político, sociedade e questões clássicas de pesquisa sociológica, como desigualdade social, pluralidades de avaliação e coordenação, conflito e crítica, racionalização, divisão do trabalho e sua organização, e cognição social.

O papel da sociologia na compreensão da influência da quantificação é fundamental ao examinar criticamente os pontos fortes e fracos da quantificação e suas implicações para a compreensão dos fenômenos sociais \cite{camargo2021quantificaccao}. A sociologia da quantificação fornece um ponto de vista crítico através do qual podemos entender o papel da quantificação na formação de nossa compreensão da realidade social. Essa perspectiva sobre a quantificação ressalta suas dimensões construídas e suas dimensões reais, demonstrando como as estatísticas produzem e são produzidas.

Portanto, o exame dos processos sociais e das relações de poder que sustentam a produção e o uso de números pode esclarecer as diversas maneiras pelas quais a quantificação pode iluminar ou obscurecer as realidades sociais, revelando quanto distorcendo tais fatos.

\subsection{A relação entre sociologia da quantificação e justiça}

Reiterando, a sociologia da quantificação postula que a quantificação está longe de ser neutra ou objetiva, desafiando a visão convencional dos números como imparciais e neutros. Em vez disso, destaca as relações de poder ocultas subjacentes à produção de estatísticas e o papel da sociologia em descobrir essas relações. Essa perspectiva transformou a percepção da relação entre estatística e política, revelando que essas ferramentas não são apenas instrumentos neutros de aferição. Em vez disso, elas estão profundamente arraigadas no tecido social e político da sociedade servindo como instrumentos de poder e governança \cite{turnbull2022slow}.

A sociologia da quantificação também examina criticamente como os números e dados estatísticos direcionam as políticas públicas, a pesquisa acadêmica e nossas percepções da realidade. Ela desafia a objetividade percebida de números e estatísticas, argumentando que são construções sociais profundamente enraizadas em estruturas sociais e percepções humanas. Um debate importante nesse campo gira em torno do conceito de objetividade \cite{salais2012quantificação}. 

Embora a quantificação ofereça o potencial de objetividade e precisão, ela também pode ser influenciada por preconceitos, suposições e limitações. Como já dito, estatísticas e quantificações são frequentemente usadas por atores políticos para legitimar determinadas políticas ou posições, corroborando a dinâmica de poder inerente ao processo de quantificação. Dessarte a sociologia da quantificação nos desafia a questionar as suposições que sustentam a produção e uso de números \cite{turnbull2022slow}. Essa perspectiva crítica é importante para promover uma compreensão mais matizada do papel da quantificação na sociedade, todavia, apesar dos avanços significativos feitos neste campo, ainda existem lacunas significativas na literatura, principalmente no que tange suas implicações para a justiça social numa perspectiva de informar abordagens de quantificação mais equitativas e democráticas, contribuindo para uma sociedade mais justa e inclusiva.

Outro aspecto da relação entre quantificação e justiça resvala no conceito de objetividade. Enquanto alguns argumentam que a quantificação pode fornecer conhecimento objetivo sobre o mundo, outros afirmam que todo conhecimento é socialmente construído e, portanto, inerentemente subjetivo. Essa dicotomia entre objetividade e subjetividade é um tema central no discurso sobre a quantificação \cite{desrosieres2009real}. A quantificação, em sua busca pela objetividade, busca a igualdade de tratamento e a imparcialidade nos processos dos indivíduos. Isso se baseia na premissa de que os números, sendo desprovidos de viés pessoal, podem fornecer uma representação neutra e justa da realidade. No entanto, essa perspectiva é muitas vezes contestada pelo argumento de que a quantificação, em sua essência, é uma construção social moldada pelos contextos sociais, políticos e culturais em que foi produzida e utilizada, logo, seria inerentemente subjetiva podendo, assim, potencialmente perpetuar os preconceitos daqueles que a criaram \cite{desrosieres2009real}.

Além disso, quando utilizada na esfera pública, a governança pela quantificação vê a diversidade das práticas sociais a partir de uma pretensa objetividade visando criptografar o mundo em representações padronizadas desconectadas das experiências individuais. Essa abordagem, embora aparentemente imparcial, muitas vezes desconsidera possíveis iniqüidades de tratamento da pessoa humana em processos de quantificação muitas vezes opacos em relação às suas regras de elaboração, observação e interpretação dos números. Essa falta de transparência pode prejudicar a legitimidade das instituições governamentais, especialmente quando indivíduos que acreditam ter sofrido desrespeito à sua dignidade encontram dificuldades para contestar esses instrumentos.

A crítica da quantificação vai além de sua falta de transparência. Envolve reconhecer as limitações dos fatos objetivos nos processos de quantificação. Essa crítica não visa rejeitar a quantificação, mas defender uma abordagem abrangente que incorpore outras metodologias de pesquisa e formas de quantificação mais justas para aumentar sua legitimidade na sociedade \cite{salais2016quantificação}. Objetiva também entender como a quantificação pode contribuir ou prejudicar a busca da justiça social \cite{alonso1987politics}.

\subsection{Direito como objeto quantitativo}

A quantificação, o processo de medir e expressar algo em números, tem sido cada vez mais utilizada no sistema de justiça, particularmente no âmbito das diretrizes de condenação \cite{salais2016quantification}. No entanto, essa prática pode prejudicar a busca pela justiça social, pois muitas vezes falha em dar conta das nuances e complexidades de casos individuais. No final do século 20, uma legislação foi aprovada nos Estados Unidos para padronizar as sentenças criminais. Essa foi uma tentativa de aumentar a objetividade e neutralidade no sistema de justiça americano. No entanto, a implementação desse sistema foi repleta de desafios. Após vinte anos de experiência, os juristas americanos chegaram ao consenso de que a aplicação obrigatória de um sistema de diretrizes de condenação foi um fracasso \cite{espeland2008sociology}. A condenação mecânica provou ser muito trabalhosa na prática, e a ignorância das idiossincrasias de casos concretos frequentemente produzia sentenças irracionais e injustas.

A ideologia neoliberal dominante, então, submeteu perigosamente a justiça e a sociedade estadunidense à ordem do cálculo, permitindo a reificação das pessoas por meio de sua história criminal para admitir a quantificação como norma de seu julgamento e como instrumento de seu controle. O impacto dessas ideologias dominantes na percepção da justiça e do Direito não pode ser subestimado. É crucial que cientistas e profissionais do direito questionem e critiquem o uso da quantificação na busca da justiça. Esse exame crítico da quantificação pode fornecer informações valiosas sobre como as ideologias dominantes moldam a percepção da justiça e do Direito. Entretanto, não basta apenas criticar o atual sistema mas também é necessário defender formas mais justas de quantificação para aumentar sua legitimidade na sociedade \cite{camargo2022estado}. O exemplo supracitado de padronização de sentenças criminais evidenciou como pode-se exacerbar as desigualdades sociais existentes ao afetar desproporcionalmente as comunidades marginalizadas \cite{lynch2019narrative}. Esse caso claro de como a quantificação pode minar a justiça, em vez de promovê-la, enseja o debate de diversos aspectos que serão trabalhados no próximo capítulo.

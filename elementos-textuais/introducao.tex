% ------------------------------------------------------------------------------
% Introdução
% ------------------------------------------------------------------------------

\chapter{Introdução}
\label{chap_introducao}

\begin{citacao}
Quantification is now more and more clearly seen not only as a neutral epistemic tool producing knowledge about the world, but also as a tool of power and government, a driving force in transforming the world \cite{didier2022}.
\end{citacao}
\begin{citacao}
Também em nossa época, como em todos os tempos, o fundamental no desenvolvimento do Direito não está no ato de legislar nem na jurisprudência ou na aplicação do Direito, mas na própria sociedade \cite{ehrlich1967fundamentos}. 
\end{citacao}
\begin{citacao}
Quando se faz jurimetria, estuda-se o Direito através das marcas que ele deixa na sociedade \cite{abj2022}.
\end{citacao}

Conforme \citeonline{didier2022}, supracitado, explana, a quantificação é uma ferramenta de poder e governo que produz conhecimentos sobre o mundo, bem como é uma força motriz que o transforma, afetando diversos aspectos da vida em sociedade como relacionamentos interpessoais, administração de organizações, avaliação de condutas, consolidação dos Estados modernos, evolução dos mercados financeiros, entre outros, de forma que tornou-se um expediente relevante para o desenvolvimento de outros campos do conhecimento que são umbilicalmente conectados a realidades sociais como a sociologia e o Direito \cite{camargo2021estudos}.

Como acima mencionado por \citeonline{ehrlich1967fundamentos}, o desenvolvimento do Direito baseia-se na sociedade, no conhecimento de suas realidades, conhecimento esse que pode ser obtido por meio da investigação dos modos de resolução de conflitos de suas associações sociais. Conflitos surgem do relacionamento mútuo de um conjunto de pessoas em uma certa ordem interna, definidora de direitos e deveres legais (regras jurídicas) entre elas, que, ao ser rompida, reafirma e/ou redefine essas regras através de meios institucionalizados, como o Estado \cite{ehrlich1967fundamentos}. Essa objetiva investigação para compreender o funcionamento de certa ordem jurídica é um movimento empírico que tem avançado rapidamente nas abordagens ao Direito visando a conhecer, por exemplo, os conflitos apresentados aos tribunais de justiça, a efetividade de uma lei nova, as relações de dominação, os contratos, os acordos, as associações, os testamentos, os casamentos, as decisões judiciais, as jurisprudências, entre outros \cite{ehrlich1967fundamentos} \cite{nunes2016jurimetria}.

Essas práticas concretas (contratos, sentenças, etc), que afetam associações sociais, demonstram um efetivo desenvolvimento do Direito ao concretizarem, por meio da solução de conflitos específicos, as normas jurídicas da sociedade. Esse processo de resolução de determinados problemas é influenciado não só pelas declarações de intenções de legisladores ou juristas por meio de, respectivamente, normas gerais ou individuais, mas também por valores e experiências pessoais, religião e tantos outros fatores que são objetos de estudo de um novo campo do conhecimento, a saber, a jurimetria \cite{nunes2016jurimetria}.

A jurimetria propõe-se a desenvolver o Direito estudando seu plano concreto e seus espaços institucionais de criação de normas visando a investigar, de forma indutiva, processos decisórios em que conflitos sociais são solucionados e, para tanto, utiliza a quantificação como ferramenta de abordagem a essa ciência a fim de reconhecer, caracterizar e organizar tendências e padrões jurisprudenciais, produzindo, assim, uma melhor compreensão do funcionamento, na prática, de determinada ordem jurídica \cite{nunes2016jurimetria}.

As marcas que o Direito deixa na sociedade referidas acima pela \citeonline{abj2022} não são exclusivamente as decisões judiciais e as legislações, mas são, também, outros fatos jurídicos como decisões administrativas, decretações de falências, formalização de contratos, recuperações judiciais de empresas, aumento do número de processos judiciais, quantidade de juízes por unidade federativa, etc. Fazer jurimetria, então, é aplicar métodos para estudar esses fatos jurídicos, porém não são quaisquer métodos; devem ser métodos quantitativos, a exemplo da estatística \cite{nunes2016jurimetria}.

Na obra brasileira seminal sobre essa nova área do conhecimento, \citeonline{nunes2016jurimetria} justifica apropriada a abordagem estatística a questões jurídicas explanando que o mundo social no qual o Direito fundamenta sua atuação e evolução não é determinista e, caso algum fenômeno de lá assim o fosse, o Direito só consegueria compreendê-lo por meio de observações empíricas e de inferências a partir destas. Dessarte, essa ferramenta quantitativa buscaria reconstituir um elemento de causalidade a fim de estudar (1) os diversos fatores - econômicos, valorativos, geográficos, etc - que condicionam a produção de normas, gerais ou individuais, e (2) as reações que essas normas provocam nas associações sociais \cite{nunes2016jurimetria}.

Nesse sentido, o objeto de estudo da jurimetria são as condutas, em função de normas jurídicas, de regulantes e regulados. Assim, tais normas não são propriamente o objeto de interesse, mas marcas que estabelecem momentos de decisões, registrando aspectos de seus sentidos e de suas motivações, que ajudam a entender o funcionamento de uma ordem jurídica por meio da apreensão estatística dos comportamentos coletivos em função dessas marcas \cite{nunes2016jurimetria}.

Mesmo que se consiga estudar as condutas individuais de cada sujeito de Direito integrado a certa sociedade, fazer inferências gerais a partir disso é temerário, pois os fenômenos regulares advindos dos comportamentos coletivos de agentes jurídicos são influenciados também pelo todo, pelo organismo social, que condiciona o comportamento de seus integrantes. Identificar essas influências, essas causas coletivas, pode auxiliar no entendimento da regularidade de certos padrões de conduta como fatos sociais e na previsão, por meio de modelos probabilísticos, das reações de um conjunto de pessoas em uma certa ordem jurídica frente a conflitos advindos de seu relacionamento mútuo. Devido a isso, a ferramenta quantitativa da estatística torna-se um método relevante à jurimetria para estudar, em função das normas jurídicas, condutas humanas não só individuais, mas também expressas em fatos sociais, que permeiam todo o Direito \cite{nunes2016jurimetria} \cite{ehrlich1967fundamentos}.

Verifica-se, nesse ponto, uma confluência entre as ciências sociais empíricas da sociologia e da jurimetria em que ambas utilizam-se da descrição de regularidades de comportamentos sociais para investigar as causas coletivas desses fenômenos padronizados \cite{van2006facts}. Uma das principais fontes que evidenciam comportamentos sociais para a jurimetria é o Judiciário, compreendido como um gerador de dados que apresentam o funcionamento completo da ordem jurídica, a exemplo do judiciário brasileiro, referido como \emph{pré-sal sociológico} por \citeonline{nunes2016jurimetria}, que pode ser compreendido como um quase-experimento devido a sua variabilidade e distribuição de materiais vastos para serem analisados através da sociologia e da estatística.

Outra aproximação dessas duas ciências sociais está na relevância que deram a práticas de quantificação. Enquanto a jurimetria as utilizou como ferramentas metodológicas de abordagem ao Direito, a sociologia as compreendeu como objeto de estudo devido a seus efeitos político-sociais que podem condicionar como indivíduos percebem a realidade social e se governam por essa percepção, a exemplo da confiança e da autoridade que indicadores, porcentagens e índices públicos, que são frequentemente ressignificados, costumam inspirar \cite{camargo2021estudos} \cite{daniel2013numeros}.

O entendimento da quantificação como ferramenta não só de medição mas também de coordenação inspirou diversas pesquisas acadêmicas, com destaque aos trabalhos do estatístico, sociólogo e historiador da ciência, Alain Desrosières. Ele explanou a quantificação como sendo a síntese das ações de convir e medir, retirando a falsa ideia de neutralidade, objetividade e tecnicidade dessa ferramenta \cite{desrosieres2013pour} \cite{camargo2021estudos}.

Ademais, os efeitos condicionantes do comportamento humano entre diferentes grupos sociais, advindos da quantificação como ferramenta de coordenação, ficam mais evidentes ao serem estudados pela sociologia, sendo revelados, assim, os modos de operação das relações de poder, a exemplo da objetivação, presente em qualquer forma de contagem, que aponta certos domínios não-comensuráveis que passam a ser convencionados pelos números, dotando, assim, de escala e dimensão aquilo que só podia ser referenciado qualitativamente, permitindo que influenciem na transformação da percepção coletiva, concorrendo para mudanças sociais \cite{camargo2021quantificaccao}.

Esses efeitos também ficam mais evidentes quando é o Estado que faz uso dessa ferramenta de coordenação. O governo de um Estado visa a solucionar problemas socialmente relevantes por meio da efetivação de políticas públicas. Mas a criação de uma política pública necessita objetivar tais problemas por meio de ferramentas quantitativas a fim de delimitar os domínios das ações governamentais e de dar legibilidade a esses problemas. Dessa forma, como a utilização da quantificação implica realizar definições operatórias convencionais para depois efetuar medições, então temos que essas investigações quantitativas oficiais irão refletir a agenda político-partidária do governo em questão e, portanto, os problemas socialmente relevantes serão objetivados segundo os interesses políticos dos governantes \cite{camargo2021estudos}.

Segundo \citeonline{nunes2016jurimetria}, discussões em torno de hipóteses de política pública devem se valer de instrumentos como a jurimetria para a realização de uma abordagem empírica, cautelosa e imparcial que levante os dados necessários à (1) objetivação de normas jurídicas que orientem a solução de certos conflitos de interesse caso sejam implementadas por meio de políticas públicas jurídicas e à (2) avaliação dos estímulos produzidos através de políticas públicas, se estes conseguiram condicionar comportamentos socialmente desejados. Esse autor também explana que essa tentativa de promover mudanças comportamentais nos sujeitos de Direito é um dos objetivos mediatos da ordem jurídica que, por meio de seus operadores que criam e aplicam normas, visa a propagar atitudes socialmente desejadas efetivando, assim, seu propósito como ferramenta de controle social \cite{nunes2016jurimetria}.

Observa-se, nessas propostas da jurimetria, uma convergência de seus conceitos com aqueles relacionados à quantificação sob uma ótica sociológica, a saber, de ferramenta de coordenação que tem ingerência sobre as vidas das pessoas a fim de condicionar determinados comportamentos. Esse relacionamento evidencia que eventuais pontes interdisciplinares entre os conceitos próprios desse novo campo do conhecimento e outras áreas do saber podem ser academicamente férteis.

Nesse sentido, a relevância da quantificação para a sociologia tornou-a um objeto de estudo a título próprio, a saber, a sociologia da quantificação, que é um campo do conhecimento que vem produzindo subsídios teóricos muito interessantes para evidenciar como fatores sociais, técnicos e políticos interagem na produção dos números que se dizem relevantes para a sociedade \cite{berman2018sociology}. Essa abordagem mostra que as estatísticas resultam de práticas sociais de categorização de dimensões da realidade e que esses números produzidos podem ser repolitizados de diversas maneiras, já que há escolhas e limitações implícitas em todos procedimentos quantitativos, de forma que a neutralidade e a imparcialidade são impossíveis e sempre há uma agenda normativa subjacente \cite{camargo2021quantificaccao}.

Assim, quando surgem novos campos do saber que se utilizam de expedientes quantitativos - pretensamente neutros, objetivos e imparciais - para abordar seus objetos de estudo, então capaz que seja razoável abordá-los também por meio da sociologia da quantificação a fim de melhor compreender essa arma poderosa, a quantificação, na produção dos conhecimentos e, consequentemente, das relações de poder nesses campos emergentes \cite{didier2013pour}. Todavia, são poucos os trabalhos acadêmicos que realizam essa abordagem sociológica.

Esta pesquisa, então, caminha para suprir essa lacuna acadêmica ao buscar compreender as principais propostas da jurimetria a partir das categorias e debates da sociologia da quantificação, visando a subsidiar melhores entendimentos sobre:
\begin{itemize}
	\item como práticas quantitativas estabelecem relações de coordenação e dominação ao produzirem conhecimentos, em que definem o que pode ser visto e o que deve permanecer oculto em certos momentos, implicando, assim, transformações da ordem político-social \cite{camargo2021estudos} \cite{mennicken2019s};
	\item como essas relações de poder influenciam na constituição das subjetividades ao implementarem novos modelos de racionalidade, como os baseados na eficiência e na utilidade, que transferem o risco para os indivíduos ao quantificarem seus comportamentos não-econômicos (decisões judiciais, criminalidade, vida familiar, etc) para analisá-los economicamente, de forma a remover dessa chave analítica princípios de proteção universal, de dignidade humana, de cooperação social, de empatia e de solidariedade \cite{camargo2021estudos} \cite{camargo2021quantificaccao};
	\item como os produtores de informações quantitativas por meio da jurimetria podem aumentar sua credibilidade técnico-científica e legitimidade político-social ao reconhecerem - e refletirem sobre - os efeitos e condicionantes advindos do uso dessa ferramenta nada imparcial \cite{camargo2021quantificaccao}.
\end{itemize}


\section{Justificativa}
\label{sec_motivacao}

A importância deste trabalho está em preencher uma lacuna na literatura acadêmica em relação a como resultados de discussões acerca da quantificação como objeto de estudo da sociologia podem ser utilizados para compreender propostas que quantificam fenômenos jurídicos, como a jurimetria.

\section{Definição do problema de pesquisa}
\label{sec_definicao_problema_pesquisa}

Como compreender as principais propostas da jurimetria a partir das categorias e debates da sociologia da quantificação?

\section{Objetivo geral}
Compreender a jurimetria a partir de suas relações com a sociologia da quantificação.

\section{Objetivos específicos}
\begin{itemize}
	\item Contextualizar como o ato de quantificar tornou-se um expediente relevante para a organização da sociedade e, portanto, para a sociologia como objeto a título próprio de estudo.
	\item Descrever contornos gerais de abordagens quantitativas do Direito e, mais especificamente, as principais propostas da jurimetria e seu impacto no Brasil e no mundo visando a analisá-la a partir de contribuições da sociologia da quantificação.
\end{itemize}

\section{Delimitação do tema}
\label{sec_contribuicoes}

A abordagem dessa pesquisa ao campo do conhecimento da quantificação, onde diferentes saberes se cruzam, gira em torno de um estudo das relações da sociologia da quantificação com a jurimetria visando a melhor compreender os impactos político-sociais que a quantificação aplicada ao Direito pode causar numa sociedade, chegando-se, assim, ao seguinte tema: ESTUDO DA JURIMETRIA NUMA ABORDAGEM SOCIOLÓGICA DA QUANTIFICAÇÃO.

\section{Organização do trabalho}
\label{sec_organizacao_trabalho}

Normalmente ao final da introdução é apresentada, em um ou dois parágrafos curtos, a organização do restante do trabalho acadêmico.
Deve-se dizer o quê será apresentado em cada um dos demais capítulos.

Segue um exemplo:

Este trabalho está organizado em capítulos, incluindo o presente.
No \autoref{chap_fundamentacao_teorica} são apresentados alguns dos principais conceitos necessários que fundamentam o desenvolvimento deste trabalho.
A \hyperref[chap_trabalhos_relacionados]{revisão bibliográfica} deste trabalho apresenta uma revisão dos principais estudos relacionados ao tema, descrevendo seus resultados e suas contribuições.
Por fim, no \autoref{chap_conclusao} são apresentadas as conclusões, bem como as perspectivas de trabalhos futuros.
